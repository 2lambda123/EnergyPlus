\chapter{Internal Variables}\label{internal-variables}

Internal variables form a category of built-in data accessible for EMS. They are internal in that they access information about the input file from inside EnergyPlus. You should already have access to the information contained in these because they depend on other content in the IDF. However, it would be inconvenient to always have to coordinate changes between Erl programs and the rest of the IDF. Internal variables simplify the process of keeping an Erl program in sync with other changes to the model. These differ from the built-in variables in that they may or may not be created in every simulation and have user-defined names that distinguish among different instances of the same type of data. The internal variables differ from sensors in that they are usually static values that do not change over time. The constants might vary from run to run but never within a single run period. Internal variables are read only.

Internal variables are automatically made available whenever an input file includes basic EMS input objects and the model they are associated with is included in the input file. To use an internal variable in an Erl program, you must declare it with an EnergyManagementSystem:InternalVariable input object. This object assigns a specified Erl variable name to contain the value in an internal EnergyPlus data structure. The EDD file lists the specific internal variable types, their unique identifying names, and the units. The rest of this section provides information about specific internal variables.
