\chapter{EMS Actuators}\label{ems-actuators}

This section provides details about EMS actuators, the conduits by which Erl programs control EnergyPlus simulations. They \emph{actuate} selected features inside EnergyPlus. Rather than add a new set of controls and component models that have EMS awareness, they generally override established features.

Internal to EnergyPlus are two separate lists of actuators:~ those that are available for use, and those that are being used in the EMS. While implementing the EMS, developers have added code to control points to make them available for use in EMS. The actuators available in a given model depend on the non-EMS-related content of the input file. Similar to how EnergyPlus reports the available output variables to the RDD file, a list of available actuators is written to the EDD file (depending on the settings in Output:EnergyManagementSystem). Note that the EDD file is ONLY produced if you have set up EMS / Erl programs. To use an actuator in EMS, you need to enter an EnergyManagementSystem:Actuator input object. The EDD file contains information needed for this input object; however, it is not available until after the output has been generated from a previous run. If you do not see a particular actuator in the EDD output file, it is not available in EMS for that particular model. There is no way to create a new actuator (except by modifying EnergyPlus source code); you can only customize the behavior of current actuators. If you need an actuator that you suspect could be provided by the program but is not available, you can use the regular system(s) for user support to request it be added to the program. New actuators are added to the program from time to time so the availability of actuators will grow with each new release.

Actuators override things inside EnergyPlus whenever the actuator's Erl variable is set to a value other than ``Null.''~ Null is a special built-in variable used to deactivate an actuator. Once an actuator starts overriding, it will continue to do so until the actuator's Erl variable is set to Null. You should set the actuators to Null whenever you want the EMS to revert to normal operation.

Beginning with version 8.6, EnergyPlus checks the list of actuators that are being used in the EMS to see if they all were in fact used.  Because there is no need to declare Erl variables, it is relatively easy for a simple spelling mistake to go undetected.  If the Erl variable name for the actuator (in the IDF input object for EnergyManagementSystem:Actuator) is not exactly correct in the Erl program, then the program could be setting the value of a separate local variable rather than the intended actuator.  To help detect this, the program now checks the acutators to see if they were ever set to anything, even Null.  A warning is issued to identify each actuator that was not used. 

The rest of this section provides details about specific actuators.
