\section{Lookup Tables}\label{lookup-tables}

\subsection{Table:Lookup}\label{table-lookup}

Lookup tables provide a method to evaluate a group of data that may or
may not conform to a fundamental equation. Lookup tables can interpolate
the actual data using a linear interpolation or piecewise cubic spline
using the Btwxt library (https://github.com/bigladder/btwxt).
As with curve objects, the lookup table can be used anywhere a valid
curve object name is allowed so long as they share the same number of
independent variables. Care must be taken to ensure the table data
format is consistent with the associate model that is using the
performance curve (e.g., DX cooling coil capacity as a function of
temperature where independent variable X1 = indoor wet-bulb temperature
and independent variable X2 = outdoor dry-bulb temperature).

A lookup table can be specified to use either linear or cubic interpolation
independently for each dimension (input variable). For performance
points outside the defined grid space, an extrapolation method--constant
or linear--can be set independently for each dimension. Finally, each
axis can have specified boundaries beyond which extrapolation is not
permitted.

\subsubsection{Linear Interpolation}\label{linear-interpolation}

For linear interpolation in 1 dimension, given known values of \(f\) at
\(x_0\) and \(x_1\), and a point \(x\) between \(x_0\) and \(x_1\), the
value at \(x\) is estimated by:

\[f\left(x\right) = \left(1-\mu\right) \cdot f\left(x_0\right) + \mu \cdot f\left(x_1\right)\]

where \(\mu\) is the location of \(x\) expressed as a fraction of the
distance between \(x_0\) and \(x_1\):

\[\mu = \left(x-x_0\right) / \left(x_1 - x_0\right)\]

\subsubsection{Cubic Spline
Interpolation}\label{cubic-spline-interpolation}

The general formula for a 1-dimensional piecewise cubic spline (for
\(x\) between known values \(f\left(x_0\right)\) and
\(f\left(x_1\right)\)) is:

\[\begin{array}{rll}
    f\left(x\right) &= \left(2\mu^3 - 3\mu^2 + 1\right) \cdot f\left(x_0\right)
    &+ \left(-2\mu^3 + 3\mu^2\right) \cdot f\left(x_1\right) \\
    &+ \left(\mu^3 - 2\mu^2 + \mu\right) \cdot f^\prime \left(x_0\right)
    &+ \left(\mu^3 - \mu^2\right) \cdot f^\prime \left(x_1\right)
\end{array}\]

The Catmull-Rom cubic spline interpolation defines the derivatives
\(f^\prime \left(x_0\right)\) and \(f^\prime\left(x_1\right)\) as the
slope between the previous and following grid points on the axis:

\[\begin{array}{rl}
    f^\prime(x_0) &= ( f(x_1) - f(x_{-1}) ) / ( x_1 - x_{-1}) \\
    f^\prime(x_1) &= ( f(x_2) - f(x_0) ) / ( x_2 - x_0)
\end{array}\]

When the hypercube is at the edge of the grid, Catmull-Rom simply
extends the slope of the final segment for defining the slope terms:
i.e., if there is no \(x_{-1}\), we substitute \(x_0\) into the
\(f^\prime(x_0)\) formula.

If a mix of interpolation methods are specified among the included
dimensions, the interpolator will perform that mix as requested.
If the lookup point is beyond the grid edge on any axis, the interpolator
will perform the requested extrapolation method (linear or constant) on
that dimension while proceeding with interpolation along any in-bounds
dimension. If a ``do-not-extrapolate-beyond'' boundary is specified, the
interpolator will perform a constant extrapolation from that boundary outward
(in that dimension), and it will return the resulting numerical answer
along with a warning that the request was outside the boundaries.

