\section{Whole-Facility Electric Service}\label{section-whole-facility-electric-service}
-----------------------------------------

Whenever an EnergyPlus model includes anything that consumes electricity the program will do some basic modeling of what can be thought of as the main panel where utility grid connects to the building and HVAC loads.  The following output variables and meters are created and filled to report on the overall, whole-facility power flows at the main panel: 

\begin{tabular}{c|c|c}
  $ {P_{purch}}  $ &    Facility Total Purchased Electric Power & [W] \tabularnewline
  $ {Q_{purch}}  $ &    Facility Total Purchased Electric Energy & [J] \tabularnewline
  $ {P_{net,purch}}  $ &    Facility Net Purchased Electric Power & [W] \tabularnewline
  $ {Q_{net,purch}}  $ &    Facility Net Purchased Electric Energy & [J] \tabularnewline
  $ {P_{surp}}  $ &    Facility Total Surplus Electric Power & [W] \tabularnewline
  $ {Q_{surp}}  $ &    Facility Total Surplus Electric Energy & [J] \tabularnewline
  $ {P_{bldg,dmd}}  $ &    Facility Total Building Electric Demand Power & [W] \tabularnewline
  $ {P_{HVAC,dmd}}  $ &    Facility Total HVAC Electric Demand Power & [W] \tabularnewline
  $ {P_{dmd}}  $ &    Facility Total Electric Demand Power & [W] \tabularnewline
  $ {P_{prod}}  $ &    Facility Total Produced Electric Power & [W] \tabularnewline
  $ {Q_{prod}}  $ &   Facility Total Produced Electric Energy & [J]
\end{tabular}

The models for electric power service use the meter reporting system inside EnergyPlus to obtain data from all the other models in EnergyPlus for the electric power demand (and maybe power production).  The ElectricPowerServiceManager uses a query routine called "GetInstantMeterValue" which returns the sum for all the metered equipment for the specified meter.  Values for   $P_{bldg,dmd}$ and $P_{HVAC,dmd}$ are both filled from the "Electricity:Facility" meter.  $P_{bldg,dmd}$ is filled by passing an argument that triggers to return the "Building" electric load, which is at the zone timestep, while $P_{HVAC,dmd}$ is filled by passing an argument that returns the "HVAC" electric load, which is at the system timestep.  The modeling is based on average power levels for the duration of the timestep. The return values are converted from energy in Joules to power in Watts by dividing by the number of seconds in the zone timestep for building loads and by the system timestep for the HVAC loads.  Then the total facility power demand is the sum of the building loads and the HVAC loads. 

\begin{equation}
  P_{dmd} = P_{bldg,dmd} + P_{HVAC,dmd}
\end{equation}

When the electric power service device models described below are being used, they also make use of the meter reporting system using the components of the ElectricityProduced:Facility ( ResourceType=ElectricityProduced) meter to determine aggregate on-site electricity production rate, $P_{prod}$.  This value rolls up all the power generated and consumed by the on-site generators, storage, and power conversion equipment:

\begin{itemize}
  \item $P_{prod,cogen}$ is the power from fuel-fired generators, from meter Cogeneration:ElectricityProduced.
  \item $P_{prod,pv}$ is the power from solar photovoltaics, from meter Photovoltaic:ElectricityProduced.
  \item $P_{prod,wind}$ is the power from wind generators, from meter WindTurbine:ElectricityProduced.
  \item $P_{prod,storage}$ is the power from electric storage, from meter ElectricStorage:ElectricityProduced, with storage charging negative and storage discharging positive.
  \item $P_{prod,conversion}$ is the power loss by converting between different voltages or AC and DC, from meter PowerConversion:ElectricityProduced which are metered as negative values that decrease the amount of electricity produced.
\end{itemize}

\begin{equation}
  {P_{prod}}= {P_{prod,cogen}} + {P_{prod,pv}} + {P_{prod,wind}} + {P_{prod,storage}} + {P_{prod,conversion}}
\end{equation}

Then the net purchased electricity, which can be either positive or negative (for export), is calculated using,

\begin{equation}
  {P_{net,purch}} = {P_{dmd}} - {P_{prod}}
\end{equation}

Only the positive portion of the net purchased power is the purchased power.

\begin{equation}
  {P_{purch}} = { max ( {P_{net,purch}}, 0.0 ) }
\end{equation}

The surplus electric power is exported out to the grid and is only positive. 

\begin{equation}
  {P_{surp}} =  { max ( ({P_{prod}} - {P_{dmd}}) , 0.0 ) }
\end{equation}

The energy versions of the power variables are calculated by multiplying the power level by the number of seconds in the system timestep. 