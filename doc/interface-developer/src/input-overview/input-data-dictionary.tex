\section{Input Data Dictionary}\label{input-data-dictionary}

The input data dictionary specifies the ``definitions'' for each line that will be processed in the input data file.

Structure in the input data dictionary allows for descriptions that may be useful for interface developers.~ The Input Processor ignores everything but the essentials for getting the ``right stuff'' into the program.~ Developers have been (and will continue to be) encouraged to include comments and other documentation in the IDD.

Internal to the data dictionary (using special ``comment'' characters) is a structured set of conventions for including information on each object.~ This is shown in section on Input Details below.

\subsection{Rules specific to the Input Data Dictionary}\label{rules-specific-to-the-input-data-dictionary}

In addition to the rules for both files (listed above), the IDD also has the limitation:

\begin{itemize}
\tightlist
\item
  Duplicate Section names and Duplicate Class names are not allowed.~ That is, the first class of an item named X will be the one used during processing.~ Error messages will appear if you try to duplicate definitions.
\end{itemize}
