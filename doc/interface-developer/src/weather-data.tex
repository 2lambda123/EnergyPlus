\chapter{Weather Data}\label{weather-data}

Weather data in EnergyPlus is a simple text-based format, similar to the input data and output data files. The weather data format includes basic location information in the first eight lines: location (name, state/province/region, country), data source, latitude, longitude, time zone, elevation, peak heating and cooling design conditions, holidays, daylight saving period, typical and extreme periods, two lines for comments, and period covered by the data. The data are also comma-separated and contain much of the same data in the TMY2 weather data set (NREL 1995). EnergyPlus does not require a full year or 8760 (or 8784) hours in its weather files. In fact, EnergyPlus allows and reads subsets of years and even sub-hourly (5 minute, 15 minute) data---the weather format includes a `minutes' field. EnergyPlus comes with a utility that reads standard weather service file types such as TMY2, IWEC and WYEC2 files, as examples, as well as being able to read a user defined custom format.

The ``data dictionary'' for EnergyPlus Weather Data is shown in the Auxiliary Programs document -- please review that document for further information.
