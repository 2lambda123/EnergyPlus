\section{Introduction to Economics}\label{introduction-to-economics}

While it is true that monthly utility bills are often directly related to monthly energy consumption and monthly peak demand, due to the elaborate regulatory environment and the changing value of the energy based on load factor, the calculations involved are often complicated. The EnergyPlus approach is to allow you to model the individual component charges that make up the utility bill and report the results on a monthly basis.

To show you how simple the specification of a rate can be here is a simple example for a flat charge of 8 cents for every kWh sold.

\begin{lstlisting}

UtilityCost:Tariff,
    TariffExample1,          ! Name
    Electricity:Facility,    ! Output Meter Name
    kWh;               ! Conversion Factor Choice


  UtilityCost:Charge:Simple,
    ChargeExample1,          ! Charge Variable Name
    TariffExample1,          ! Tariff Name
    totalEnergy,             ! Source Variable
    Annual,                   ! Season
    EnergyCharges,           ! Category Variable Name
    0.08;                    ! Cost Per Unit Value or Variable Name
\end{lstlisting}

Overall, the goals of the economics section of EnergyPlus are to make simple tariffs easy to model and complex tariffs possible to model.

The cost implications of alternative, energy-related design options may be important in determining the whether or not a particular design is viable. Although the main purpose of EnergyPlus is to determine the energy performance of a building, component cost modeling is provided to allow estimating initial construction costs associated with the building and HVAC system being modeled. This is provided in order to ensure that cost estimates are consistent with the EnergyPlus model and it allows using results calculated during a simulation such as equipment sizes and surface areas. The cost estimating capabilities are not comprehensive and are not intended to duplicate all the types of analyses included in cost estimating software. Economics modeling is being implemented in stages. Future versions of EnergyPlus will add other investment related analyses such as life-cycle cost. The objects used by EnergyPlus for calculating first costs for construction projects are as follows:

\begin{itemize}
\item
  ComponentCost:Adjustments
\item
  ComponentCost:Reference
\item
  ComponentCost:LineItem
\end{itemize}

The results of the cost estimate are reported in a Construction Cost Estimate Summary table provided that the user requests the results using the object Output:Table:SummaryReports object with the Component Cost Economics Summary option selected. The style of this predefined report can be controlled using the OutputControl:Table:Style object. If the Annual Building Performance Report Summary is also requested, then cost estimate results will follow that data in the same output file.

The rest of this Economics section is organized to first present reference material on tariff calculations followed by construction cost estimating.
