\section{Group -- Humidifiers}\label{group-humidifiers}

\subsection{Humidifier:Steam:Electric}\label{humidifiersteamelectric}

The electric steam humidifier is a component that represents an electrically heated, self contained steam humidifier. The component uses electrical energy to convert ordinary tap water to steam which it then injects into the supply air stream by means of a blower fan. The actual unit might be an electrode-type humidifier or a resistance-type humidifier.

The humidifier model includes local control of the humidifier unit to meet a humidity ratio setpoint on its air outlet node. A set point manager is needed to put a setpoint on the exit node but no other local controllers are needed. The humidifier will add moisture to meet the humidity ratio setpoint.

\subsubsection{Inputs}\label{inputs-023}

\paragraph{Field: Name}\label{field-name-022}

A unique user assigned name for a particular humidifier unit. Any reference to this unit by another object will use this name.

\paragraph{Field: Availability Schedule Name}\label{field-availability-schedule-name-010}

The name of the schedule (ref: Schedule) that denotes whether the unit can run during a given time period. A schedule value of 0 indicates that the unit is off for that time period. A schedule value greater than 0 indicates that the unit can operate during the time period. If this field is blank, the schedule has values of 1 for all time periods.

\paragraph{Field: Rated Capacity}\label{field-rated-capacity-000}

The nominal full output water addition rate of the unit in m\(^{3}\)/sec of water at 5.05 C. This field is autosizable.

\paragraph{Field: Rated Power}\label{field-rated-power-000}

The nominal full output power consumption of the unit in watts, exclusive of the blower fan power consumption and any standby power. This field can be autosized. When it is autosized, its calculated from the rated capacity in kg/s and the enthalpy rise in J/kg of the feed water from the a reference temperature of liquid water at 20 °C to a saturated steam at 100 °C.

\paragraph{Field: Rated Fan Power}\label{field-rated-fan-power}

The nominal full output power consumption of the blower fan in watts.

\paragraph{Field: Standby Power}\label{field-standby-power-000}

The standby power consumption in watts. This amount of power will be consumed whenever the unit is available (as defined by the availability schedule).

\paragraph{Field: Air Inlet Node Name}\label{field-air-inlet-node-name-004}

The name of the HVAC system node from which the unit draws inlet air.

\paragraph{Field: Air Outlet Node Name}\label{field-air-outlet-node-name-004}

The name of the HVAC system node to which the unit sends its outlet air.

\paragraph{Field: Water Storage Tank Name}\label{field-water-storage-tank-name}

This field is optional. If left blank or omitted, then the humidifier obtains its water directly from the mains water. If the name of a Water Storage Tank is specified, then the humidifier will try to obtain its water from that tank. If the tank can t provide all the water then the rest will be drawn from the mains and the humidifier will still operate.

An IDF example:

\begin{lstlisting}

Humidifier:Steam:Electric,
  Humidifier 1,                   !- Name
  FanAndCoilAvailSched,   !- Availability Schedule Name
  0.00000379,                       !- Rated Capacity {m3/s}
  10200.,                               !- Rated Power {W}
  27.,                                     !- Rated Fan Power {W}
  2.,                                       !- Standby Power {W}
  Cooling Coil Air Outlet Node, !- Air Inlet Node Name
  Air Loop Outlet Node;   !- Air Outlet Node Name
\end{lstlisting}

\subsubsection{Outputs}\label{outputs-016}

\begin{itemize}
\item
  HVAC,Average,Humidifier Water Volume Flow Rate {[}m3/s{]}
\item
  HVAC,Sum,Humidifier Water Volume{[}m3{]}
\item
  HVAC,Average,Humidifier Electricity Rate {[}W{]}
\item
  HVAC,Sum,Humidifier Electricity Energy {[}J{]}
\item
  Zone,Meter,Humidifier:Water {[}m3{]}
\item
  Zone,Meter,Humidifier:Electricity {[}J{]}
\item
  HVAC,Average,Humidifier Storage Tank Water Volume Flow Rate {[}m3/s{]}
\item
  HVAC,Sum,Humidifier Storage Tank Water Volume {[}m3{]}
\item
  HVAC,Average,Humidifier Starved Storage Tank Water Volume Flow Rate {[}m3/s{]}
\item
  HVAC,Sum,Humidifier Starved Storage Tank Water Volume {[}m3{]}
\item
  Zone,Meter,Humidifier:MainsWater {[}m3{]}
\item
  HVAC,Sum,Humidifier Mains Water Volume {[}m3{]}
\end{itemize}

\paragraph{Humidifier Water Volume Flow Rate {[}m3/s{]}}\label{humidifier-water-volume-flow-rate-m3s}

This field reports the water consumption rate of the steam humidifier in cubic meters of water per second.

\paragraph{Humidifier Water Volume{[}m3{]}}\label{humidifier-water-volumem3}

This ouput is the cubic meters of water consumed by the steam humidifier over the timestep being reported.

\paragraph{Humidifier Electricity Rate {[}W{]}}\label{humidifier-electric-powerw}

This output is the electricity consumption rate in Watts of the steam humidifier.

\paragraph{Humidifier Electricity Energy {[}J{]}}\label{humidifier-electric-energy-j}

This is the electricity consumption in Joules of the steam humidifier over the timestep being reported.

\paragraph{Humidifier:Water {[}m3{]}}\label{humidifierwater-m3}

This meter output contains the sum of the water consumed (in cubic neters of water during the report timestep) by all the steam humidifiers at the HVAC level in the simulation.

\paragraph{Humidifier:Electricity {[}J{]}}\label{humidifierelectricity-j}

This meter output contains the sum of the electricity consumed (in Joules during the report timestep) by all the steam humidifiers at the HVAC level in the simulation.

\paragraph{Humidifier Storage Tank Water Volume Flow Rate {[}m3/s{]}}\label{humidifier-storage-tank-water-volume-flow-rate-m3s}

\paragraph{Humidifier Storage Tank Water Volume {[}m3{]}}\label{humidifier-storage-tank-water-volume-m3}

These outputs contain the rate and volume of water obtained from water storage tank. These are only present if the humidifier is connected to a Water Storage Tank for its water supply.

\paragraph{Humidifier Starved Storage Tank Water Volume Flow Rate {[}m3/s{]}}\label{humidifier-starved-storage-tank-water-volume-flow-rate-m3s}

\paragraph{Humidifier Starved Storage Tank Water Volume {[}m3{]}}\label{humidifier-starved-storage-tank-water-volume-m3}

These outputs contain the rate and volume of water that could not be obtained from the water storage tank. The component will still operate as if it did get all the water with the balance obtained directly from the mains

\paragraph{Humidifier Mains Water Volume {[}m3{]}}\label{humidifier-mains-water-volume-m3}

This output contains the volume of water obtained from the mains.

\subsection{Humidifier:Steam:Gas}\label{humidifiersteamgas}

The gas fired steam humidifier is a component that represents a gas fired self-contained steam humidifier. The component uses gas fired energy to convert ordinary tap water to steam which it then blows or injects into the supply air stream. Blower fan may not be required depending on how the dry steam is delivered into the supply air stream. The humidifier model includes local control of the humidifier unit to meet a humidity ratio setpoint on its air outlet node of the unit. A humidity set point manager is needed to put a setpoint on the outlet node but no other local controllers are needed. The humidifier either blows or injects dry steam to meet the humidity ratio setpoint requirement. If the Rated Gas Use Rate input field is not autosized, the thermal efficiency input specified will be ignored and ovverriden by a thermal efficiency value determined from user specified Rated Gas Use Rate, rated capacity (m3/s) and design conditions for sizing calculation.

\subsubsection{Inputs}\label{inputs-1-021}

\paragraph{Field: Name}\label{field-name-1-020}

A unique user assigned name for a particular humidifier unit. Any reference to this unit by another object will use this name.

\paragraph{Field: Availability Schedule Name}\label{field-availability-schedule-name-1-008}

The name of the schedule (ref: Schedule) that denotes whether the unit can run during a given time period. A schedule value of 0 indicates that the unit is off for that time period. A schedule value greater than 0 indicates that the unit can operate during the time period. If this field is blank, the schedule has values of 1 for all time periods.

\paragraph{Field: Rated Capacity}\label{field-rated-capacity-1}

The nominal full capacity water addition rate in m3/s of water at 5.05 C.

\paragraph{Field: Rated Gas Use Rate \{W\}}\label{field-rated-gas-use-rate-w}

The nominal gas use rate in Watts. This input field can be autosized. When this input field is autosized, it is calculated from the rated capacity in kg/s, the enthalpy rise in J/kg of the feed water from a reference temperature of liquid water at 20 °C to a saturated steam at 100 °C and user specified thermal efficiency. If this input field is hardsized and the Inlet Water Temperature Option input field is selected as FixedInletWaterTemperature, then the thermal efficiency input field will not be used in the calculation or else if the Inlet Water Temperature Option input selected is VariableInletWaterTemperature, then the user specified thermal efficiency value will be overridden using internally calculated efficiency from the capacity, rated gas use rate and design condition.

\paragraph{Field: Thermal Efficiency}\label{field-thermal-efficiency}

The thermal efficiency of the gas fired humidifier. The thermal efficiency is based on the higher heating value of the fuel. The default value is 0.8. If Rated Gas Use Rate in the field above is not autosized and the Inlet Water Temperature Option input field selected is FixedInletWaterTemperature, then the thermal efficiency specified will be ignored in the calculation, or else if the Inlet Water Temperature Option input field is specified as VariableInletWaterTemperature, then the user specified thermal efficiency value will be overridden using internally calculated matching the capacity, rated gas use rate specified and design condition defined for sizing calculation.

\paragraph{Field: Thermal Efficiency Modifier Curve Name}\label{field-thermal-efficiency-modifier-curve-name}

This is thermal efficiency modifier curve name of unit. This curve is normalized, i.e., the curve output value at rated condition is 1.0. If this input field is blank, then constant efficiency value specified in the input field above will be used. Allowed thermal efficiency modifier curve types are linear, quadratic, or cubic. These curves are solely a function of part load ratio.

\paragraph{Field: Rated Fan Power}\label{field-rated-fan-power-1}

The nominal full capacity electric power input to the blower fan in Watts. If no blower fan is required to inject the dry steam to the supply air stream, then this input field is set to zero.

\paragraph{Field: Auxiliary Electric Power}\label{field-auxiliary-electric-power}

The auxiliary electric power input in watts. This amount of power will be consumed whenever the unit is available (as defined by the availability schedule). This electric power is used for control purpose only.

\paragraph{Field: Air Inlet Node Name}\label{field-air-inlet-node-name-1-004}

The name of the HVAC system node from which the unit draws inlet air.

\paragraph{Field: Air Outlet Node Name}\label{field-air-outlet-node-name-1-003}

The name of the HVAC system node to which the unit sends its outlet air.

\paragraph{Field: Water Storage Tank Name}\label{field-water-storage-tank-name-1}

This field is optional. If left blank or omitted, then the humidifier obtains its water directly from the mains water. If the name of a Water Storage Tank is specified, then the humidifier will try to obtain its water from that tank. If the tank can t provide all the water then the rest will be drawn from the mains and the humidifier will still operate.

\paragraph{Field: Inlet Water Temperature Option}\label{field-inlet-water-temperature-option}

This field is a key/choice field that tells which humidifier water inlet temperature to use: fixed inlet temperature or variable water inlet temperature that depends on the source. Currently allowed water sources are main water or water storage tank in water use objects. The key/choice are: FixedInletWaterTemperature, with this choice, the gas fired humidifier will use a fixed 20C water inlet temperature. VariableInletWaterTemperature, with this choice, the gas fired humidifier will use water inlet temperature that depends on the source temperature. If a water use storage tank name is specified, then the gas humidifier water inlet temperature will be the storage water temperature, or else it uses water main temperature. The default main water temperature is 10 °C. If left blank or omitted, then the humidifier assumes fixed inlet water temperature of 20 °C.

An IDF example:

\begin{lstlisting}

Humidifier:Steam:Gas,
  Main Gas Humidifier,!- Name
  ALWAYS_ON,          !- Availability Schedule Name
  4.00E-5,            !- Rated Capacity {m3/s}
  104000,             !- Rated Gas Use Rate {W}
  1.0,                !- Thermal Efficiency {-}
  ,                   !- Thermal Efficiency Modifier Curve Name
  0,                  !- Rated Fan Power {W}
  0,                  !- Auxiliary Electric Power {W}
  Mixed Air Node 1,   !- Air Inlet Node Name
  Main Humidifier Outlet Node,  !- Air Outlet Node Name
  ;                   !- Water Storage Tank Name
\end{lstlisting}

Steam Gas Humidifier Outputs

\begin{itemize}
\tightlist
\item
  HVAC,Average,Humidifier Water Volume Flow Rate {[}m3/s{]}
\item
  HVAC,Sum,Humidifier Water Volume{[}m3{]}
\item
  HVAC,Average,Humidifier Gas Use Rate{[}W{]}
\item
  HVAC,Sum,Humidifier Gas Use Energy {[}J{]}
\item
  HVAC,Average,Humidifier Auxiliary Electricity Rate {[}W{]}
\item
  HVAC,Sum,Humidifier Auxiliary Electricity Energy {[}J{]}
\item
  HVAC,Meter,Humidifier:Water {[}m3{]}
\item
  HVAC,Meter,Humidifier:Gas {[}J{]}
\item
  HVAC,Meter,Humidifier:Electricity {[}J{]}
\item
  HVAC,Average,Humidifier Storage Tank Water Volume Flow Rate {[}m3/s{]}
\item
  HVAC,Sum,Humidifier Storage Tank Water Volume {[}m3{]}
\item
  HVAC,Average,Humidifier Starved Storage Tank Water Volume Flow Rate {[}m3/s{]}
\item
  HVAC,Sum,Humidifier Starved Storage Tank Water Volume {[}m3{]}
\item
  Zone,Meter,Humidifier:MainsWater {[}m3{]}
\item
  HVAC,Sum,Humidifier Mains Water Volume {[}m3{]}
\end{itemize}

\paragraph{Humidifier Water Volume Flow Rate {[}m3/s{]}}\label{humidifier-water-volume-flow-rate-m3s-1}

This field reports the water consumption rate of the steam humidifier in cubic meters of water per second.

\paragraph{Humidifier Water Volume {[}m3{]}}\label{humidifier-water-volume-m3}

This output is the cubic meters of water consumed by the steam humidifier over the timestep being reported.

\paragraph{Humidifier Gas Use Rate {[}W{]}}\label{humidifier-gas-use-rate-w}

This output is the gas use rate of the gas fired steam humidifier in Watts.

\paragraph{Humidifier Gas Use Energy {[}J{]}}\label{humidifier-gas-use-energy-j}

This output is the gas consumption of the gas fired steam humidifier in Joules.

\paragraph{Humidifier Auxiliary Electricity Rate {[}W{]}}\label{humidifier-auxiliary-electric-power-w}

This output is the auxiliary electricity consumption rate in Watts of the gas fired steam humidifier. This is the auxiliary electric power input to the blower fan and control unit.

\paragraph{Humidifier Auxiliary Electricity Energy {[}J{]}}\label{humidifier-auxiliary-electric-energy-j}

This is the auxiliary electricity consumption in Joules of the gas fired steam humidifier over the timestep being reported. This is the auxiliary electric energy consumed by the blower fan and control unit. This auxiliary electric energy is reported meter output Humidifier:Electricity.

\paragraph{Humidifier:Water {[}m3{]}}\label{humidifierwater-m3-1}

This meter output contains the sum of the water consumed (in cubic meters of water during the report timestep) by all the steam humidifiers at the HVAC level in the simulation.

\paragraph{Humidifier:Gas {[}J{]}}\label{humidifiergas-j}

This meter output contains the sum of the gas consumed (in Joules during the report timestep) by all the steam humidifiers at the HVAC level in the simulation.

\paragraph{Humidifier Storage Tank Water Volume Flow Rate {[}m3/s{]}}\label{humidifier-storage-tank-water-volume-flow-rate-m3s-1}

\paragraph{Humidifier Storage Tank Water Volume {[}m3{]}}\label{humidifier-storage-tank-water-volume-m3-1}

These outputs contain the rate and volume of water obtained from water storage tank. These are only present if the humidifier is connected to a Water Storage Tank for its water supply.

\paragraph{Humidifier Starved Storage Tank Water Volume Flow Rate {[}m3/s{]}}\label{humidifier-starved-storage-tank-water-volume-flow-rate-m3s-1}

\paragraph{Humidifier Starved Storage Tank Water Volume {[}m3{]}}\label{humidifier-starved-storage-tank-water-volume-m3-1}

These outputs contain the rate and volume of water that could not be obtained from the water storage tank. The component will still operate as if it did get all the water with the balance obtained directly from the mains

\paragraph{Humidifier Mains Water Volume {[}m3{]}}\label{humidifier-mains-water-volume-m3-1}

This output contains the volume of water obtained from the mains.