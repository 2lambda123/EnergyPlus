\section{Group Performance Tables}\label{group-performance-tables}

This group of objects consists of tabular data which are used to characterize the performance of HVAC equipment. The use of performance tables eliminate the need to perform a regression analysis to calculate the performance curve equation coefficients. The following descriptions define the input requirements for the performance table objects.

\subsection{Table:Lookup}\label{tablelookup}

Input for tables representing data with one or more independent
variables.

\subsubsection{Inputs}\label{inputs}

\paragraph{Field: Name}\label{field-name}

A unique user-assigned name for an instance of a lookup table. When a
table is used, it is referenced by this name. The name of this table
object may be used anywhere a valid performance curve object is allowed.

\paragraph{Field: Independent Variable List
Name}\label{field-independent-variable-list-name}

The name of a Table:IndependentVariableList object that defines the
independent variables that comprise the dimensions of the tabular data.

\paragraph{Field: Normalization Method}\label{field-normalization-method}

Determines if and how this output will be normalized. Choices are:

\begin{enumerate}
  \item \emph{No}: Do not normalize output.
  \item \emph{DivisorOnly}: Normalize the output using only the value entered into the Normalization Divisor field, below.
  \item \emph{AutomaticWithDivisor}: Normalize the output by dividing the output data by the value of the table at each of the independent variable normalization reference values in the associated independent variable list AND dividing by the value entered into the Normalization Divisor field, below.
\end{enumerate}

Normalization applies to the Output Value fields as well as the Minimum
and Maximum Output fields.

\paragraph{Field: Normalization Divisor}\label{field-normalization-divisor}

Defines a value by which output values are normalized if the Normalization Method
field is set to \emph{DivisorOnly} or \emph{AutomaticWithDivisor}. Default is 1.0.

\paragraph{Field: Minimum Output}\label{field-minimum-output}

The (non-normalized) minimum allowable value of the evaluated table after interpolation
and extrapolation. Values less than the minimum will be replaced by the
minimum. If this field is left blank, no limit is imposed.

\paragraph{Field: Maximum Output}\label{field-maximum-output}

The (non-normalized) maximum allowable value of the evaluated table after interpolation
and extrapolation. Values greater than the maximum will be replaced by
the maximum. If this field is left blank, no limit is imposed.

\paragraph{Field: Output Unit Type}\label{field-output-unit-type}

This field is used to indicate the kind of units that may be associated
with the output values. It is used by interfaces or input editors (e.g., IDF Editor) to display the
appropriate SI and IP units for the Minimum Output and Maximum Output.
If none of these options are appropriate, select \emph{Dimensionless}
which will have no unit conversion. Options are:

\begin{itemize}
  \tightlist
  \item
  \emph{Dimensionless}
  \item
  \emph{Capacity}
  \item
  \emph{Power}
\end{itemize}

\paragraph{Field: External File
Name}\label{field-external-file-name}

The name of an external CSV file that represents the tabular data. This
file should be formatted such that the data for this particular output
is ordered according to the order of the corresponding independent
variables. For example, for three independent variables (\texttt{iv1},
\texttt{iv2}, \texttt{iv3}) with 3, 2, and 4 values respectively. The
output values (\texttt{out{[}iv1{]}{[}iv2{]}{[}iv3{]}}) should be
ordered as:

\begin{longtable}[]{@{}llll@{}}
\toprule
\texttt{iv1} & \texttt{iv2} & \texttt{iv3} &
\texttt{output}\tabularnewline
\midrule
\endhead
\texttt{iv1{[}1{]}} & \texttt{iv2{[}1{]}} & \texttt{iv3{[}1{]}} &
\texttt{out{[}1{]}{[}1{]}{[}1{]}}\tabularnewline
\texttt{iv1{[}1{]}} & \texttt{iv2{[}1{]}} & \texttt{iv3{[}2{]}} &
\texttt{out{[}1{]}{[}1{]}{[}2{]}}\tabularnewline
\texttt{iv1{[}1{]}} & \texttt{iv2{[}1{]}} & \texttt{iv3{[}3{]}} &
\texttt{out{[}1{]}{[}1{]}{[}3{]}}\tabularnewline
\texttt{iv1{[}1{]}} & \texttt{iv2{[}1{]}} & \texttt{iv3{[}4{]}} &
\texttt{out{[}1{]}{[}1{]}{[}4{]}}\tabularnewline
\texttt{iv1{[}1{]}} & \texttt{iv2{[}2{]}} & \texttt{iv3{[}1{]}} &
\texttt{out{[}1{]}{[}2{]}{[}1{]}}\tabularnewline
\texttt{iv1{[}1{]}} & \texttt{iv2{[}2{]}} & \texttt{iv3{[}2{]}} &
\texttt{out{[}1{]}{[}2{]}{[}2{]}}\tabularnewline
\texttt{iv1{[}1{]}} & \texttt{iv2{[}2{]}} & \texttt{iv3{[}3{]}} &
\texttt{out{[}1{]}{[}2{]}{[}3{]}}\tabularnewline
\texttt{iv1{[}1{]}} & \texttt{iv2{[}2{]}} & \texttt{iv3{[}4{]}} &
\texttt{out{[}1{]}{[}2{]}{[}4{]}}\tabularnewline
\texttt{iv1{[}2{]}} & \texttt{iv2{[}1{]}} & \texttt{iv3{[}1{]}} &
\texttt{out{[}2{]}{[}1{]}{[}1{]}}\tabularnewline
\texttt{iv1{[}2{]}} & \texttt{iv2{[}1{]}} & \texttt{iv3{[}2{]}} &
\texttt{out{[}2{]}{[}1{]}{[}2{]}}\tabularnewline
\texttt{iv1{[}2{]}} & \texttt{iv2{[}1{]}} & \texttt{iv3{[}3{]}} &
\texttt{out{[}2{]}{[}1{]}{[}3{]}}\tabularnewline
\texttt{iv1{[}2{]}} & \texttt{iv2{[}1{]}} & \texttt{iv3{[}4{]}} &
\texttt{out{[}2{]}{[}1{]}{[}4{]}}\tabularnewline
\texttt{iv1{[}2{]}} & \texttt{iv2{[}2{]}} & \texttt{iv3{[}1{]}} &
\texttt{out{[}2{]}{[}2{]}{[}1{]}}\tabularnewline
\texttt{iv1{[}2{]}} & \texttt{iv2{[}2{]}} & \texttt{iv3{[}2{]}} &
\texttt{out{[}2{]}{[}2{]}{[}2{]}}\tabularnewline
\texttt{iv1{[}2{]}} & \texttt{iv2{[}2{]}} & \texttt{iv3{[}3{]}} &
\texttt{out{[}2{]}{[}2{]}{[}3{]}}\tabularnewline
\texttt{iv1{[}2{]}} & \texttt{iv2{[}2{]}} & \texttt{iv3{[}4{]}} &
\texttt{out{[}2{]}{[}2{]}{[}4{]}}\tabularnewline
\texttt{iv1{[}3{]}} & \texttt{iv2{[}1{]}} & \texttt{iv3{[}1{]}} &
\texttt{out{[}3{]}{[}1{]}{[}1{]}}\tabularnewline
\texttt{iv1{[}3{]}} & \texttt{iv2{[}1{]}} & \texttt{iv3{[}2{]}} &
\texttt{out{[}3{]}{[}1{]}{[}2{]}}\tabularnewline
\texttt{iv1{[}3{]}} & \texttt{iv2{[}1{]}} & \texttt{iv3{[}3{]}} &
\texttt{out{[}3{]}{[}1{]}{[}3{]}}\tabularnewline
\texttt{iv1{[}3{]}} & \texttt{iv2{[}1{]}} & \texttt{iv3{[}4{]}} &
\texttt{out{[}3{]}{[}1{]}{[}4{]}}\tabularnewline
\texttt{iv1{[}3{]}} & \texttt{iv2{[}2{]}} & \texttt{iv3{[}1{]}} &
\texttt{out{[}3{]}{[}2{]}{[}1{]}}\tabularnewline
\texttt{iv1{[}3{]}} & \texttt{iv2{[}2{]}} & \texttt{iv3{[}2{]}} &
\texttt{out{[}3{]}{[}2{]}{[}2{]}}\tabularnewline
\texttt{iv1{[}3{]}} & \texttt{iv2{[}2{]}} & \texttt{iv3{[}3{]}} &
\texttt{out{[}3{]}{[}2{]}{[}3{]}}\tabularnewline
\texttt{iv1{[}3{]}} & \texttt{iv2{[}2{]}} & \texttt{iv3{[}4{]}} &
\texttt{out{[}3{]}{[}2{]}{[}4{]}}\tabularnewline
\bottomrule
\end{longtable}

Alternatively, the output may be defined using the \emph{Output Value
\textless{}x\textgreater{}} fields below.

\paragraph{Field: External File Column
Number}\label{field-external-file-column-number}

The column number (starting at 1) in the CSV file corresponding to this
output.

\paragraph{Field: External File Starting Row
Number}\label{field-external-file-starting-row-number}

The row number (starting at 1) in the CSV file where the data for this
output begins. If there are not enough rows of data to fill out the full
grid of data an error will be presented to the user.

\paragraph{Output Value
\textless{}x\textgreater{}}\label{output-value-x}

If not reading from an external file, this field is repeated to capture
the full set of output data in the table (if not otherwise defined in an
external file). The data for this particular output is ordered according
to the order of the corresponding independent variables. For example,
for three independent variables (\texttt{iv1}, \texttt{iv2},
\texttt{iv3}) with 3, 2, and 4 values respectively. The output values
(\texttt{out{[}iv1{]}{[}iv2{]}{[}iv3{]}}) should be ordered as:

\begin{longtable}[]{@{}llll@{}}
\toprule
\texttt{iv1} & \texttt{iv2} & \texttt{iv3} &
\texttt{output}\tabularnewline
\midrule
\endhead
\texttt{iv1{[}1{]}} & \texttt{iv2{[}1{]}} & \texttt{iv3{[}1{]}} &
\texttt{out{[}1{]}{[}1{]}{[}1{]}}\tabularnewline
\texttt{iv1{[}1{]}} & \texttt{iv2{[}1{]}} & \texttt{iv3{[}2{]}} &
\texttt{out{[}1{]}{[}1{]}{[}2{]}}\tabularnewline
\texttt{iv1{[}1{]}} & \texttt{iv2{[}1{]}} & \texttt{iv3{[}3{]}} &
\texttt{out{[}1{]}{[}1{]}{[}3{]}}\tabularnewline
\texttt{iv1{[}1{]}} & \texttt{iv2{[}1{]}} & \texttt{iv3{[}4{]}} &
\texttt{out{[}1{]}{[}1{]}{[}4{]}}\tabularnewline
\texttt{iv1{[}1{]}} & \texttt{iv2{[}2{]}} & \texttt{iv3{[}1{]}} &
\texttt{out{[}1{]}{[}2{]}{[}1{]}}\tabularnewline
\texttt{iv1{[}1{]}} & \texttt{iv2{[}2{]}} & \texttt{iv3{[}2{]}} &
\texttt{out{[}1{]}{[}2{]}{[}2{]}}\tabularnewline
\texttt{iv1{[}1{]}} & \texttt{iv2{[}2{]}} & \texttt{iv3{[}3{]}} &
\texttt{out{[}1{]}{[}2{]}{[}3{]}}\tabularnewline
\texttt{iv1{[}1{]}} & \texttt{iv2{[}2{]}} & \texttt{iv3{[}4{]}} &
\texttt{out{[}1{]}{[}2{]}{[}4{]}}\tabularnewline
\texttt{iv1{[}2{]}} & \texttt{iv2{[}1{]}} & \texttt{iv3{[}1{]}} &
\texttt{out{[}2{]}{[}1{]}{[}1{]}}\tabularnewline
\texttt{iv1{[}2{]}} & \texttt{iv2{[}1{]}} & \texttt{iv3{[}2{]}} &
\texttt{out{[}2{]}{[}1{]}{[}2{]}}\tabularnewline
\texttt{iv1{[}2{]}} & \texttt{iv2{[}1{]}} & \texttt{iv3{[}3{]}} &
\texttt{out{[}2{]}{[}1{]}{[}3{]}}\tabularnewline
\texttt{iv1{[}2{]}} & \texttt{iv2{[}1{]}} & \texttt{iv3{[}4{]}} &
\texttt{out{[}2{]}{[}1{]}{[}4{]}}\tabularnewline
\texttt{iv1{[}2{]}} & \texttt{iv2{[}2{]}} & \texttt{iv3{[}1{]}} &
\texttt{out{[}2{]}{[}2{]}{[}1{]}}\tabularnewline
\texttt{iv1{[}2{]}} & \texttt{iv2{[}2{]}} & \texttt{iv3{[}2{]}} &
\texttt{out{[}2{]}{[}2{]}{[}2{]}}\tabularnewline
\texttt{iv1{[}2{]}} & \texttt{iv2{[}2{]}} & \texttt{iv3{[}3{]}} &
\texttt{out{[}2{]}{[}2{]}{[}3{]}}\tabularnewline
\texttt{iv1{[}2{]}} & \texttt{iv2{[}2{]}} & \texttt{iv3{[}4{]}} &
\texttt{out{[}2{]}{[}2{]}{[}4{]}}\tabularnewline
\texttt{iv1{[}3{]}} & \texttt{iv2{[}1{]}} & \texttt{iv3{[}1{]}} &
\texttt{out{[}3{]}{[}1{]}{[}1{]}}\tabularnewline
\texttt{iv1{[}3{]}} & \texttt{iv2{[}1{]}} & \texttt{iv3{[}2{]}} &
\texttt{out{[}3{]}{[}1{]}{[}2{]}}\tabularnewline
\texttt{iv1{[}3{]}} & \texttt{iv2{[}1{]}} & \texttt{iv3{[}3{]}} &
\texttt{out{[}3{]}{[}1{]}{[}3{]}}\tabularnewline
\texttt{iv1{[}3{]}} & \texttt{iv2{[}1{]}} & \texttt{iv3{[}4{]}} &
\texttt{out{[}3{]}{[}1{]}{[}4{]}}\tabularnewline
\texttt{iv1{[}3{]}} & \texttt{iv2{[}2{]}} & \texttt{iv3{[}1{]}} &
\texttt{out{[}3{]}{[}2{]}{[}1{]}}\tabularnewline
\texttt{iv1{[}3{]}} & \texttt{iv2{[}2{]}} & \texttt{iv3{[}2{]}} &
\texttt{out{[}3{]}{[}2{]}{[}2{]}}\tabularnewline
\texttt{iv1{[}3{]}} & \texttt{iv2{[}2{]}} & \texttt{iv3{[}3{]}} &
\texttt{out{[}3{]}{[}2{]}{[}3{]}}\tabularnewline
\texttt{iv1{[}3{]}} & \texttt{iv2{[}2{]}} & \texttt{iv3{[}4{]}} &
\texttt{out{[}3{]}{[}2{]}{[}4{]}}\tabularnewline
\bottomrule
\end{longtable}

\subsubsection{Example IDF}\label{example-idf}

\begin{lstlisting}
Table:Lookup,
  HPACCoolCapFT,           !- Name
  HPACCoolCapFT_IndependentVariableList,  !- Independent Variable List Name
  AutomaticWithDivisor,    !- Normalization Method
  ,                        !- Normalization Divisor
  0,                       !- Minimum Output
  40000,                   !- Maximum Output
  Dimensionless,           !- Output Unit Type
  ,                        !- External File Name
  ,                        !- External File Column Number
  ,                        !- External File Starting Row Number
  24421.69383, 22779.73113, 21147.21662, 19794.00525, 19524.15032, 18178.81244, 16810.36004,
  25997.3589,  24352.1562,  22716.4017,  21360.49033, 21090.0954,  19742.05753, 18370.84513,
  28392.31868, 26742.74198, 25102.61348, 23743.0571,  23471.93318, 22120.2503,  20745.3119,
  29655.22876, 28003.546,   26361.31143, 25000,       24728.52506, 23375.08713, 21998.35468,
  31094.97495, 29441.02425, 27796.52175, 26433.32038, 26161.46745, 24806.13958, 23427.47518,
  33988.3473,  32330.1846,  30681.4701,  29314.75872, 29042.2038,  27683.36592, 26301.11353;
\end{lstlisting}

\subsubsection{Outputs}\label{outputs}

\paragraph{Performance Curve Output Value
{[}{]}}\label{performance-curve-output-value}

The current value of the performance table. Performance curves and
tables use the same output variable. This value is averaged over the
time step being reported. Inactive or unused performance curves will
show a value of -999 (e.g., equipment is off, a specific performance
curve is not required for this aspect of the equipment model at this
time step, etc.). This value means that the performance curve was not
called during the simulation and, therefore, not evaluated. This
inactive state value is only set at the beginning of each environment.
When averaging over long periods of time, this inactive state value may
skew results. In this case, use a detailed reporting frequency (ref.
Output:Variable object) to view results at each HVAC time step.

\paragraph{Performance Curve Input Variable
\textless{}x\textgreater{} Value
{[}{]}}\label{performance-curve-input-variable-x-value}

\subsection{Table:IndependentVariableList}\label{tableindependentvariablelist}

A list of Table:IndependentVariable references that define the size and
dimensions of the data for one or more Table:Lookup objects. The order
of this list defines the order that the tabular data must be defined.
The output values in the Table:Lookup associated with this list will
list the output in an order cycling through the last item in the list
first, and then the second to last, and so on with the the first item
cycling last. For example, for three independent variables
(\texttt{iv1}, \texttt{iv2}, \texttt{iv3}) with 3, 2, and 4 values
respectively. The output values
(\texttt{out{[}iv1{]}{[}iv2{]}{[}iv3{]}}) should be ordered as:

\begin{longtable}[]{@{}llll@{}}
\toprule
\texttt{iv1} & \texttt{iv2} & \texttt{iv3} &
\texttt{output}\tabularnewline
\midrule
\endhead
\texttt{iv1{[}1{]}} & \texttt{iv2{[}1{]}} & \texttt{iv3{[}1{]}} &
\texttt{out{[}1{]}{[}1{]}{[}1{]}}\tabularnewline
\texttt{iv1{[}1{]}} & \texttt{iv2{[}1{]}} & \texttt{iv3{[}2{]}} &
\texttt{out{[}1{]}{[}1{]}{[}2{]}}\tabularnewline
\texttt{iv1{[}1{]}} & \texttt{iv2{[}1{]}} & \texttt{iv3{[}3{]}} &
\texttt{out{[}1{]}{[}1{]}{[}3{]}}\tabularnewline
\texttt{iv1{[}1{]}} & \texttt{iv2{[}1{]}} & \texttt{iv3{[}4{]}} &
\texttt{out{[}1{]}{[}1{]}{[}4{]}}\tabularnewline
\texttt{iv1{[}1{]}} & \texttt{iv2{[}2{]}} & \texttt{iv3{[}1{]}} &
\texttt{out{[}1{]}{[}2{]}{[}1{]}}\tabularnewline
\texttt{iv1{[}1{]}} & \texttt{iv2{[}2{]}} & \texttt{iv3{[}2{]}} &
\texttt{out{[}1{]}{[}2{]}{[}2{]}}\tabularnewline
\texttt{iv1{[}1{]}} & \texttt{iv2{[}2{]}} & \texttt{iv3{[}3{]}} &
\texttt{out{[}1{]}{[}2{]}{[}3{]}}\tabularnewline
\texttt{iv1{[}1{]}} & \texttt{iv2{[}2{]}} & \texttt{iv3{[}4{]}} &
\texttt{out{[}1{]}{[}2{]}{[}4{]}}\tabularnewline
\texttt{iv1{[}2{]}} & \texttt{iv2{[}1{]}} & \texttt{iv3{[}1{]}} &
\texttt{out{[}2{]}{[}1{]}{[}1{]}}\tabularnewline
\texttt{iv1{[}2{]}} & \texttt{iv2{[}1{]}} & \texttt{iv3{[}2{]}} &
\texttt{out{[}2{]}{[}1{]}{[}2{]}}\tabularnewline
\texttt{iv1{[}2{]}} & \texttt{iv2{[}1{]}} & \texttt{iv3{[}3{]}} &
\texttt{out{[}2{]}{[}1{]}{[}3{]}}\tabularnewline
\texttt{iv1{[}2{]}} & \texttt{iv2{[}1{]}} & \texttt{iv3{[}4{]}} &
\texttt{out{[}2{]}{[}1{]}{[}4{]}}\tabularnewline
\texttt{iv1{[}2{]}} & \texttt{iv2{[}2{]}} & \texttt{iv3{[}1{]}} &
\texttt{out{[}2{]}{[}2{]}{[}1{]}}\tabularnewline
\texttt{iv1{[}2{]}} & \texttt{iv2{[}2{]}} & \texttt{iv3{[}2{]}} &
\texttt{out{[}2{]}{[}2{]}{[}2{]}}\tabularnewline
\texttt{iv1{[}2{]}} & \texttt{iv2{[}2{]}} & \texttt{iv3{[}3{]}} &
\texttt{out{[}2{]}{[}2{]}{[}3{]}}\tabularnewline
\texttt{iv1{[}2{]}} & \texttt{iv2{[}2{]}} & \texttt{iv3{[}4{]}} &
\texttt{out{[}2{]}{[}2{]}{[}4{]}}\tabularnewline
\texttt{iv1{[}3{]}} & \texttt{iv2{[}1{]}} & \texttt{iv3{[}1{]}} &
\texttt{out{[}3{]}{[}1{]}{[}1{]}}\tabularnewline
\texttt{iv1{[}3{]}} & \texttt{iv2{[}1{]}} & \texttt{iv3{[}2{]}} &
\texttt{out{[}3{]}{[}1{]}{[}2{]}}\tabularnewline
\texttt{iv1{[}3{]}} & \texttt{iv2{[}1{]}} & \texttt{iv3{[}3{]}} &
\texttt{out{[}3{]}{[}1{]}{[}3{]}}\tabularnewline
\texttt{iv1{[}3{]}} & \texttt{iv2{[}1{]}} & \texttt{iv3{[}4{]}} &
\texttt{out{[}3{]}{[}1{]}{[}4{]}}\tabularnewline
\texttt{iv1{[}3{]}} & \texttt{iv2{[}2{]}} & \texttt{iv3{[}1{]}} &
\texttt{out{[}3{]}{[}2{]}{[}1{]}}\tabularnewline
\texttt{iv1{[}3{]}} & \texttt{iv2{[}2{]}} & \texttt{iv3{[}2{]}} &
\texttt{out{[}3{]}{[}2{]}{[}2{]}}\tabularnewline
\texttt{iv1{[}3{]}} & \texttt{iv2{[}2{]}} & \texttt{iv3{[}3{]}} &
\texttt{out{[}3{]}{[}2{]}{[}3{]}}\tabularnewline
\texttt{iv1{[}3{]}} & \texttt{iv2{[}2{]}} & \texttt{iv3{[}4{]}} &
\texttt{out{[}3{]}{[}2{]}{[}4{]}}\tabularnewline
\bottomrule
\end{longtable}

\subsubsection{Inputs}\label{inputs-1}

\paragraph{Field: Name}\label{field-name-1}

A unique user-assigned name for a list of independent variables. This
name is referenced by Table:Lookup objects. The name of this list object
may be referenced by any number of Table:Lookup objects.

\paragraph{Field: Independent Variable \textless{}x\textgreater{}
Name}\label{field-independent-variable-x-name}

This field is repeated for the number of independent variables that
define the dimensions of any corresponding Table:Lookup objects that
refer to this list. Each instance provides the name of a
Table:IndependentVariable object that defines the values and properties
of an independent variable.

\subsubsection{Example IDF}\label{example-idf-1}

\begin{lstlisting}
Table:IndependentVariableList,
  HPACCoolCapFT_IndependentVariableList,  !- Name
  HPACCoolCapFT_Tewb,                     !- Independent Variable 1 Name
  HPACCoolCapFT_Todb;                     !- Independent Variable 2 Name
\end{lstlisting}

\subsection{Table:IndependentVariable}\label{tableindependentvariable}

Independent variables are used to define the size and dimensions of a
Table:Lookup object.

\subsubsection{Inputs}\label{inputs-2}

\paragraph{Field: Name}\label{field-name-2}

A unique user-assigned name for an independent variables. This name is
referenced by Table:IndependentVariableList objects. The name of this
object may be referenced by any number of Table:IndependentVariableList
objects.

\paragraph{Field: Interpolation
Method}\label{field-interpolation-method}

Method used to determine the value of the table within the bounds of its
independent variables. The choices are:

\begin{itemize}
\tightlist
\item
\emph{Linear}
\item
\emph{Cubic}
\end{itemize}

\paragraph{Field: Extrapolation
Method}\label{field-extrapolation-method}

Method used to determine the value of the table beyond the bounds of its
independent variables. The choices are:

\begin{itemize}
\tightlist
\item
\emph{Constant}: Value is the same as the interpolated value at the
closest point along the table's boundary.
\item
\emph{Linear}: Value is linearly extrapolated in all dimensions from
the interpolated value at the closest point along the table's
boundary.
%\item
%\emph{Unavailable}: The object using this curve is assumed to be not
%operable outside of the provided range of values. This will override
%any corresponding availability schedules of applicable equipment.
\end{itemize}

\paragraph{Field: Minimum Value}\label{field-minimum-value}

The minimum allowable value. Table:Lookup output for values between this
value and the lowest value provided for this independent variable will
be extrapolated according to the extrapolation method. Below this value,
extrapolation is held constant and a warning will be issued. If
extrapolation method is ``Unavailable'', the corresponding equipment
will be disabled for all values less than the lowest independent
variable value, regardless of the minimum value set here.

\paragraph{Field: Maximum Value}\label{field-maximum-value}

The maximum allowable value. Table:Lookup output for values between this
value and the highest value provided for this independent variable will
be extrapolated according to the extrapolation method. Above this value,
extrapolation is held constant and a warning will be issued. If
extrapolation method is ``Unavailable'', the corresponding equipment
will be disabled for all values greater than the highest independent
variable value, regardless of the maximum value set here.

\paragraph{Field: Normalization Reference Value}\label{field-normalization--reference-value}

The value of this independent variable where nominal or rated output is
defined. This will be used to normalize the data so that the outputs of
any Table:Lookup at this value (and the corresponding Normalization
Reference Values of the other independent variables described in the same
Table:IndependentVariableList object) are equal to 1.0. If left blank, no
normalization will be calculated. If this field is left blank the Normalization
Reference Value of all other Table:IndependentVariable objects within the same
Table:IndependentVariableList must also be left blank.

\paragraph{Field: Unit Type}\label{field-unit-type}

This field is used to indicate the kind of units that may be associated
with this independent variable. It is used by interfaces or input editors (e.g., IDF Editor) to display the
appropriate SI and IP units for the Minimum Value and Maximum Value. The
available options are shown below. If none of these options are
appropriate, select \emph{Dimensionless} which will have no unit
conversion.

\begin{itemize}
\tightlist
\item
Dimensionless
\item
Temperature
\item
VolumetricFlow
\item
MassFlow
\item
Distance
\item
Power
\end{itemize}

\paragraph{Field: External File
Name}\label{field-external-file-name-1}

The name of an external CSV file that represents the tabular data. This
file should be formatted such that the data for any output is ordered
according to the order of the corresponding independent variables. For
example, for three independent variables (\texttt{iv1}, \texttt{iv2},
\texttt{iv3}) with 3, 2, and 4 values respectively. The output values
(\texttt{out{[}iv1{]}{[}iv2{]}{[}iv3{]}}) should be ordered as:

\begin{longtable}[]{@{}llll@{}}
\toprule
\texttt{iv1} & \texttt{iv2} & \texttt{iv3} &
\texttt{output}\tabularnewline
\midrule
\endhead
\texttt{iv1{[}1{]}} & \texttt{iv2{[}1{]}} & \texttt{iv3{[}1{]}} &
\texttt{out{[}1{]}{[}1{]}{[}1{]}}\tabularnewline
\texttt{iv1{[}1{]}} & \texttt{iv2{[}1{]}} & \texttt{iv3{[}2{]}} &
\texttt{out{[}1{]}{[}1{]}{[}2{]}}\tabularnewline
\texttt{iv1{[}1{]}} & \texttt{iv2{[}1{]}} & \texttt{iv3{[}3{]}} &
\texttt{out{[}1{]}{[}1{]}{[}3{]}}\tabularnewline
\texttt{iv1{[}1{]}} & \texttt{iv2{[}1{]}} & \texttt{iv3{[}4{]}} &
\texttt{out{[}1{]}{[}1{]}{[}4{]}}\tabularnewline
\texttt{iv1{[}1{]}} & \texttt{iv2{[}2{]}} & \texttt{iv3{[}1{]}} &
\texttt{out{[}1{]}{[}2{]}{[}1{]}}\tabularnewline
\texttt{iv1{[}1{]}} & \texttt{iv2{[}2{]}} & \texttt{iv3{[}2{]}} &
\texttt{out{[}1{]}{[}2{]}{[}2{]}}\tabularnewline
\texttt{iv1{[}1{]}} & \texttt{iv2{[}2{]}} & \texttt{iv3{[}3{]}} &
\texttt{out{[}1{]}{[}2{]}{[}3{]}}\tabularnewline
\texttt{iv1{[}1{]}} & \texttt{iv2{[}2{]}} & \texttt{iv3{[}4{]}} &
\texttt{out{[}1{]}{[}2{]}{[}4{]}}\tabularnewline
\texttt{iv1{[}2{]}} & \texttt{iv2{[}1{]}} & \texttt{iv3{[}1{]}} &
\texttt{out{[}2{]}{[}1{]}{[}1{]}}\tabularnewline
\texttt{iv1{[}2{]}} & \texttt{iv2{[}1{]}} & \texttt{iv3{[}2{]}} &
\texttt{out{[}2{]}{[}1{]}{[}2{]}}\tabularnewline
\texttt{iv1{[}2{]}} & \texttt{iv2{[}1{]}} & \texttt{iv3{[}3{]}} &
\texttt{out{[}2{]}{[}1{]}{[}3{]}}\tabularnewline
\texttt{iv1{[}2{]}} & \texttt{iv2{[}1{]}} & \texttt{iv3{[}4{]}} &
\texttt{out{[}2{]}{[}1{]}{[}4{]}}\tabularnewline
\texttt{iv1{[}2{]}} & \texttt{iv2{[}2{]}} & \texttt{iv3{[}1{]}} &
\texttt{out{[}2{]}{[}2{]}{[}1{]}}\tabularnewline
\texttt{iv1{[}2{]}} & \texttt{iv2{[}2{]}} & \texttt{iv3{[}2{]}} &
\texttt{out{[}2{]}{[}2{]}{[}2{]}}\tabularnewline
\texttt{iv1{[}2{]}} & \texttt{iv2{[}2{]}} & \texttt{iv3{[}3{]}} &
\texttt{out{[}2{]}{[}2{]}{[}3{]}}\tabularnewline
\texttt{iv1{[}2{]}} & \texttt{iv2{[}2{]}} & \texttt{iv3{[}4{]}} &
\texttt{out{[}2{]}{[}2{]}{[}4{]}}\tabularnewline
\texttt{iv1{[}3{]}} & \texttt{iv2{[}1{]}} & \texttt{iv3{[}1{]}} &
\texttt{out{[}3{]}{[}1{]}{[}1{]}}\tabularnewline
\texttt{iv1{[}3{]}} & \texttt{iv2{[}1{]}} & \texttt{iv3{[}2{]}} &
\texttt{out{[}3{]}{[}1{]}{[}2{]}}\tabularnewline
\texttt{iv1{[}3{]}} & \texttt{iv2{[}1{]}} & \texttt{iv3{[}3{]}} &
\texttt{out{[}3{]}{[}1{]}{[}3{]}}\tabularnewline
\texttt{iv1{[}3{]}} & \texttt{iv2{[}1{]}} & \texttt{iv3{[}4{]}} &
\texttt{out{[}3{]}{[}1{]}{[}4{]}}\tabularnewline
\texttt{iv1{[}3{]}} & \texttt{iv2{[}2{]}} & \texttt{iv3{[}1{]}} &
\texttt{out{[}3{]}{[}2{]}{[}1{]}}\tabularnewline
\texttt{iv1{[}3{]}} & \texttt{iv2{[}2{]}} & \texttt{iv3{[}2{]}} &
\texttt{out{[}3{]}{[}2{]}{[}2{]}}\tabularnewline
\texttt{iv1{[}3{]}} & \texttt{iv2{[}2{]}} & \texttt{iv3{[}3{]}} &
\texttt{out{[}3{]}{[}2{]}{[}3{]}}\tabularnewline
\texttt{iv1{[}3{]}} & \texttt{iv2{[}2{]}} & \texttt{iv3{[}4{]}} &
\texttt{out{[}3{]}{[}2{]}{[}4{]}}\tabularnewline
\bottomrule
\end{longtable}

Independent variable values must appear in \textbf{ascending} order (an
error will be issued if this is not the case).

Alternatively, the independent variables may be defined using the
\emph{Value \textless{}x\textgreater{}} fields below.

\paragraph{Field: External File Column
Number}\label{field-external-file-column-number-1}

The column number (starting at 1) in the CSV file corresponding to this
independent variable. As the values of the independent variables each
repeat over a defined cycle, EnergyPlus will only read unique values
from this column. EnergyPlus does not validate that the cycles are
repeating correctly. In fact, the same data can be read by only defining
each value once as it is first encountered:

\begin{longtable}[]{@{}llll@{}}
\toprule
\texttt{iv1} & \texttt{iv2} & \texttt{iv3} &
\texttt{output}\tabularnewline
\midrule
\endhead
\texttt{iv1{[}1{]}} & \texttt{iv2{[}1{]}} & \texttt{iv3{[}1{]}} &
\texttt{out{[}1{]}{[}1{]}{[}1{]}}\tabularnewline
& & \texttt{iv3{[}2{]}} &
\texttt{out{[}1{]}{[}1{]}{[}2{]}}\tabularnewline
& & \texttt{iv3{[}3{]}} &
\texttt{out{[}1{]}{[}1{]}{[}3{]}}\tabularnewline
& & \texttt{iv3{[}4{]}} &
\texttt{out{[}1{]}{[}1{]}{[}4{]}}\tabularnewline
& \texttt{iv2{[}2{]}} & &
\texttt{out{[}1{]}{[}2{]}{[}1{]}}\tabularnewline
& & & \texttt{out{[}1{]}{[}2{]}{[}2{]}}\tabularnewline
& & & \texttt{out{[}1{]}{[}2{]}{[}3{]}}\tabularnewline
& & & \texttt{out{[}1{]}{[}2{]}{[}4{]}}\tabularnewline
\texttt{iv1{[}2{]}} & & &
\texttt{out{[}2{]}{[}1{]}{[}1{]}}\tabularnewline
& & & \texttt{out{[}2{]}{[}1{]}{[}2{]}}\tabularnewline
& & & \texttt{out{[}2{]}{[}1{]}{[}3{]}}\tabularnewline
& & & \texttt{out{[}2{]}{[}1{]}{[}4{]}}\tabularnewline
& & & \texttt{out{[}2{]}{[}2{]}{[}1{]}}\tabularnewline
& & & \texttt{out{[}2{]}{[}2{]}{[}2{]}}\tabularnewline
& & & \texttt{out{[}2{]}{[}2{]}{[}3{]}}\tabularnewline
& & & \texttt{out{[}2{]}{[}2{]}{[}4{]}}\tabularnewline
\texttt{iv1{[}3{]}} & & &
\texttt{out{[}3{]}{[}1{]}{[}1{]}}\tabularnewline
& & & \texttt{out{[}3{]}{[}1{]}{[}2{]}}\tabularnewline
& & & \texttt{out{[}3{]}{[}1{]}{[}3{]}}\tabularnewline
& & & \texttt{out{[}3{]}{[}1{]}{[}4{]}}\tabularnewline
& & & \texttt{out{[}3{]}{[}2{]}{[}1{]}}\tabularnewline
& & & \texttt{out{[}3{]}{[}2{]}{[}2{]}}\tabularnewline
& & & \texttt{out{[}3{]}{[}2{]}{[}3{]}}\tabularnewline
& & & \texttt{out{[}3{]}{[}2{]}{[}4{]}}\tabularnewline
\bottomrule
\end{longtable}

\paragraph{Field: External File Starting Row
Number}\label{field-external-file-starting-row-number-1}

The row number (starting at 1) in the CSV file where the data for this
independent variable begins. Any values in the same column below this
row are considered part of the range.

\paragraph{Field: Value
\textless{}x\textgreater{}}\label{field-value-x}

If not reading from an external file, this field is repeated to capture
the full set of values for this independent variable. These values must
be defined in \textbf{ascending} order (an error will be issued if this
is not the case).

\subsubsection{Example IDF}\label{example-idf-2}

\begin{lstlisting}
Table:IndependentVariable,
  HPACCoolCapFT_Tewb,      !- Name
  Cubic,                   !- Interpolation Method
  Constant,                !- Extrapolation Method
  12.77778,                !- Minimum Value
  23.88889,                !- Maximum Value
  19.44449,                !- Normalization Reference Value
  Temperature,             !- Unit Type
  ,                        !- External File Name
  ,                        !- External File Column Number
  ,                        !- External File Starting Row Number
  12.77778,                !- Value 1
  15.00000,                !- Value 2
  18.00000,                !- Value 3
  19.44449,                !- Value 4
  21.00000,                !- Value 5
  23.88889;                !- Value 6
\end{lstlisting}
