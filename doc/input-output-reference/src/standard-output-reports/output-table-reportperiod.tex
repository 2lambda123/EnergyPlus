\section{Output:Table:ReportPeriod}\label{outputtablereportperiod}

The Output:Table:ReportPeriod object allows the user to create a series of
tabular reports over a certain subset of a simulation run period. Multiple such
reporting periods may be defined, each with a unique ``Name''. Currently, it only
supports resilience reporting tables.

When it is defined, a series of reporting-period-specific resilience summary
reports will be generated at the end of all tabular reports. The results will be
based on aggregations over the intersection between the start date to the end
date defined here and a run period. Multiple reporting periods may be input.

If the report is used for the evaluation of the thermal resilience properties
during a power outage, the model configuration needs to consider common
behavioral adaptations like increased air exchange from opening windows during
hot events, or the use of alternative heating fuels (fireplaces, stoves) during
cold events. Without taking these into considerations, the model output might
be misleading.

In each table, a row corresponds to a zone or space. Four additional rows are
appended to the bottom of each table to provide summary statistics (min, max,
average, and sum) across zones/spaces in a building. Each column is a reporting
variable.

\subsection{Inputs}\label{inputs-074}

\paragraph{Field: Name}\label{field-name-066}

This required field allows the ReportPeriod to be named and will appear in the
header at the beginning of a series of resilience tables for this reporting
period.

\paragraph{Field: Report Name}\label{report-name-1}

This field specifies the group of reports to generate for this reporting period.
Currently, only ThermalResilienceSummary, CO2ResilienceSummary,
VisualResilienceSummary, and AllResilienceSummaries are allowed. It could be
extended to other tables in the future.

\paragraph{Field: Begin Year}\label{field-begin-year-2}

This optional numeric field may contain the beginning year for the range. If not
specified, the year will be zero, and will not be used in the determination of the
reporting period.

\paragraph{Field: Begin Month}\label{field-begin-month-3}

This numeric field should contain the starting month number (1 = January, 2 =
February, etc.) for the reporting period desired.

\paragraph{Field: Begin Day of Month}\label{field-begin-day-of-month-3}

This numeric field should contain the starting day of the starting month (must
be valid for month) for the reporting period desired.

\paragraph{Field: Begin Hour of Day}\label{field-begin-hour-of-day}

This numeric field should contain the starting hour of day for the reporting
period desired.

\paragraph{Field: End Year}\label{field-end-year-2}

This optional numeric field may contain the end year for the range. If not
specified, the year will be zero, and will not be used in the determination of the
reporting period.

\paragraph{Field: End Month}\label{field-end-month-3}

This numeric field should contain the ending month number (1 = January, 2 =
February, etc.) for the reporting period desired.

\paragraph{Field: End Day of Month}\label{field-end-day-of-month-3}

This numeric field should contain the ending day of the ending month (must be
valid for month) for the reporting period desired.

\paragraph{Field: End Hour of Day}\label{field-end-hour-of-day}

This numeric field should contain the end hour of day for the reporting period
desired.

The following is an example of this object

\begin{lstlisting}

Output:Table:ReportPeriod,
 ThermalResilienceReportTimeWinter,  !- field Name,
 ThermalResilienceSummary,           !- field Report Name,
 ,                                   !- Begin Year
 1,                                  !- Begin Month
 1,                                  !- Begin Day of Month
 8,                                  !- Begin Hour of Day
 ,                                   !- End Year
 1,                                  !- End Month
 3,                                  !- End Day of Month
 18;                                 !- End Hour of Day

\end{lstlisting}
