\section{Slab Program Output Files}\label{slab-program-output-files}

The following output files are created by the Slab program and saved in the output file path specified in the RunSlab.bat file:

*\_slab.ger - Error file. Input errors are reported here.

*\_slab.out - Summary of inputs, location data, and grid coordinates

*\_slab.gtp - Monthly ground temperatures and EnergyPlus idf objects

\subsection{EnergyPlus idf Objects from Slab Program}\label{energyplus-idf-objects-from-slab-program}

If the objects are placed in the normal EnergyPlus input IDF file using the ``GroundHeatTransfer:Slab:'' prefix, then the values resulting from the Slab preprocessor will be automatically included in the simulation at run time. The surfaces can reference these values using Outside Boundary Conditions of:

\begin{itemize}
\item
  GroundSlabPreprocessorAverage
\item
  GroundSlabPreprocessorCore
\item
  GroundSlabPreprocessorPerimeter
\end{itemize}

The objects that support this include:

\begin{itemize}
\item
  BuildingSurface:Detailed
\item
  Wall:Detailed
\item
  RoofCeiling:Detailed
\item
  Floor:Detailed
\end{itemize}

The slab program is capable of supplying the EnergyPlus idf objects needed to use the slab program outputs directly by copying the objects into the EnergyPlus idf file. The file containing these objects has the extension gtp, and an example is shown below. It contains the output temperatures and heat fluxes, and in addition it contains an OtherSideCoefficient object example as mentioned above, and three compact schedule objects that can be used to describe the average, core and perimeter temperatures directly in EnergyPlus. The Name in the compact schedule corresponds to the GroundTemperatureScheduleName in the OtherSideCoefficient object.

= = = = = = = = = = = = = = = = = = = = = = = = = = = = = = = = = = = = = = = = = = = = = = = =

CHECK CONVERGENCE MESSAGE AT END OF THIS FILE!

= = = = = = = = = = = = = = = = = = = = = = = = = = = = = = = = = = = = = = = = = = = = = = = =

\begin{lstlisting}
   Monthly Slab Outside Face Temperatures, C and Heat Fluxes(loss), W/(m^2)
 Perimeter Area: 304.00  Core Area: 1296.00
       Month   TAverage   TPerimeter    TCore      TInside AverageFlux PerimeterFlux CoreFlux
          1      17.74     16.41        18.05       18.00       0.88        5.39       -0.17
          2      17.49     16.15        17.81       18.00       1.73        6.29        0.66
          3      17.45     16.23        17.74       18.00       1.86        6.02        0.88
          4      18.96     17.86        19.22       20.00       3.51        7.24        2.64
          5      19.22     18.22        19.45       20.00       2.66        6.04        1.86
          6      19.28     18.38        19.49       20.00       2.44        5.48        1.73
          7      20.83     19.98        21.03       22.00       3.96        6.87        3.28
          8      21.12     20.39        21.29       22.00       2.98        5.47        2.40
          9      21.18     20.46        21.35       22.00       2.76        5.22        2.19
         10      21.17     20.23        21.39       22.00       2.82        6.02        2.08
         11      19.64     18.63        19.88       20.00       1.22        4.63        0.42
         12      19.36     18.14        19.65       20.00       2.16        6.30        1.19
\end{lstlisting}

\begin{lstlisting}
! OTHER SIDE COEFFICIENT OBJECT EXAMPLE FOR IDF FILE
 SurfaceProperty:OtherSideCoefficients,
 ExampleOSC,                !- OtherSideCoeff Name \*\*\*CHANGE THIS!\*\*\*
 0,                         !- Combined convective/radiative film coefficient
 1,                         !- N2,User selected Constant Temperature {C}
 1,                         !- Coefficient modifying the user selected constant temperature
 0,                         !- Coefficient modifying the external dry bulb temperature
 0,                         !- Coefficient modifying the ground temperature
 0,                         !- Coefficient modifying the wind speed term (s/m)
 0,                         !- Coefficient modifying the zone air temperature
                            !  part of the equation
 GroundTempCompactSchedName; !- Name of Schedule for values of const
                            ! temperature. Schedule values replace N2.
                            !  \*\*\*REPLACE WITH CORRECT NAME\*\*\*
\end{lstlisting}

\begin{lstlisting}
Schedule:Compact,
MonthlyAveSurfaceTemp, !Name
Temperature ,            !- ScheduleType
Through:   1/31,
For:AllDays,
Until:24:00,
 17.74    ,
Through:   2/28,
For:AllDays,
Until:24:00,
 17.49    ,
Through:   3/31,
For:AllDays,
Until:24:00,
 17.45    ,
Through:   4/30,
For:AllDays,
Until:24:00,
 18.96    ,
Through:   5/31,
For:AllDays,
Until:24:00,
 19.22    ,
Through:   6/30,
For:AllDays,
Until:24:00,
 19.28    ,
Through:   7/31,
For:AllDays,
Until:24:00,
 20.83    ,
Through:   8/31,
For:AllDays,
Until:24:00,
 21.12    ,
Through:   9/30,
For:AllDays,
Until:24:00,
 21.18    ,
Through:  10/31,
For:AllDays,
Until:24:00,
 21.17    ,
Through:  11/30,
For:AllDays,
Until:24:00,
 19.64    ,
Through:  12/31,
For:AllDays,
Until:24:00,
 19.36    ;
\end{lstlisting}

 - compact schedules for MonthlyPerimeterTemp and MonthlyCoreTemp are included.

\begin{lstlisting}
  Convergence has been gained.
\end{lstlisting}
