\section{eplusout.dfs}\label{eplusout.dfs}

The daylight factors report contains the pre-calculated hourly daylight factors for the four CIE sky types (clear, clear turbid, intermediate, and overcast sky), for each daylight reference points for each exterior window of a daylight enclosure which consists of one or more spaces. The file is CSV formatted -- ready for spreadsheet viewing. The input for requesting the report is simple:

\begin{lstlisting}
Output:DaylightFactors,
  SizingDays;    !- Reporting Days
\end{lstlisting}

The following shows an excerpt the daylight factors for a window without shade:

\begin{lstlisting}
This file contains daylight factors for all exterior windows of daylight zones.
MonthAndDay,Enclosure Name, Zone Name,Window Name,Window State
Hour,Reference Point,Daylight Factor for Clear Sky,Daylight Factor for Clear Turbid Sky,Daylight Factor for Intermediate Sky,Daylight Factor for Overcast Sky
01/21,Enclosure 1,ZN_1,ZN_1_WALL_NORTH_WINDOW,Base Window
1,ZN_1_DAYLREFPT1,0.00000,0.00000,0.00000,0.00000
1,ZN_1_DAYLREFPT2,0.00000,0.00000,0.00000,0.00000
1,ZN_1_DAYLREFPT3,0.00000,0.00000,0.00000,0.00000
[...]
9,ZN_1_DAYLREFPT1,2.14657E-002,1.73681E-002,1.38241E-002,1.41272E-002
9,ZN_1_DAYLREFPT2,2.14657E-002,1.73680E-002,1.38240E-002,1.41273E-002
9,ZN_1_DAYLREFPT3,2.14655E-002,1.73678E-002,1.38239E-002,1.41274E-002
[...]
24,ZN_1_DAYLREFPT1,0.00000,0.00000,0.00000,0.00000
24,ZN_1_DAYLREFPT2,0.00000,0.00000,0.00000,0.00000
24,ZN_1_DAYLREFPT3,0.00000,0.00000,0.00000,0.00000
\end{lstlisting}

The first line describes the data in the \textbf{eplusout.dfs} file.

The second line describes the header of the daily row:

{\scriptsize
\begin{longtable}[c]{>{\raggedright}p{1.0in}>{\raggedright}p{1.0in}>{\raggedright}p{2.5in}>{\raggedright}p{1.5in}}
\toprule
MonthAndDay & Zone Name & Window Name & Window State \tabularnewline
\midrule
\endfirsthead

\toprule
MonthAndDay & Zone Name & Window Name & Window State \tabularnewline
\midrule
\endhead

01/21 & ZN\_1 & ZN\_1\_WALL\_NORTH\_WINDOW & Base Window \tabularnewline
\bottomrule
\end{longtable}}

The Window State column has three possible outputs -- ``Base Window'' indicating windows without shades; a numeric value, from 0.0 to 180.0 for every 10.0 step, indicating the slat angle (in degrees) of the blind shading the window; ``Blind or Slat Applied'' indicating windows in shaded state by shades, screens, and blinds with fixed slat angle.

The lines after it describes the hourly daylight factors at each reference point:

{\scriptsize
\begin{longtable}[c]{p{0.2in}p{1.5in}p{1.0in}p{1.0in}p{1.0in}p{1.0in}}
\caption{Hourly Daylight Factors for the considered window at each reference point and 4 CIE Sky Conditions}
\label{tab:dfs-output-at-each-ref-pt}\\
\hline
Hour & Reference Point & Daylight Factor for Clear Sky & Daylight Factor for Clear Turbid Sky & Daylight Factor for Intermediate Sky & Daylight Factor for Overcast Sky \\ \hline
\endhead
%
9 & ZN\_1\_DAYLREFPT1 & 0.021466 & 0.017368 & 0.013824 & 0.014127 \\ \hline
9 & ZN\_1\_DAYLREFPT2 & 0.021466 & 0.017368 & 0.013824 & 0.014127 \\ \hline
9 & ZN\_1\_DAYLREFPT3 & 0.021465 & 0.017368 & 0.013824 & 0.014127 \\ \hline
10 & ZN\_1\_DAYLREFPT1 & 0.020245 & 0.016029 & 0.012201 & 0.014127 \\ \hline
\end{longtable}}
