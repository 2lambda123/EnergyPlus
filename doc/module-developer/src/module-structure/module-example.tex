\section{Module Example}\label{module-example}

This example can be used as a template for new HVAC component modules. In particular, the commenting structure in the module and within the subroutines should be followed closely. Of course, there is no perfect example module -- this one is particularly simple. Some others that might be examined are in files Humidifiers.f90, HVACHeatingCoils.f90 and PlantChillers.f90. Templates are also available as separate files.

In particular, the module template with routines contains structure and information pertinent to module development.

Note that in the following module, the ``Data IPShortcuts'' is not used -- rather those variables are allocated within this module -- likely because another module calls this one during input.

\textbf{Module Fans}

~ ! Module containing the fan simulation routines

~ ! MODULE INFORMATION:

~ !~~~~~~ AUTHOR~~~~~~~~ Richard J. Liesen

~ !~~~~~~ DATE WRITTEN~~ April 1998

~ !~~~~~~ MODIFIED~~~~~~ Shirey, May 2001

~ !~~~~~~ RE-ENGINEERED~ na

~ ! PURPOSE OF THIS MODULE:

~ ! To encapsulate the data and algorithms required to

~ ! manage the Fan System Component

~ ! REFERENCES: none

~ ! OTHER NOTES: none

~ ! USE STATEMENTS:

~ ! Use statements for data only modules

USE DataPrecisionGlobals

USE DataLoopNode

USE DataHVACGlobals, ONLY: TurnFansOn, TurnFansOff, Main, Cooling, Heating, Other, \&

~~~~~~~~~ OnOffFanPartLoadFraction, SmallAirVolFlow, UnbalExhMassFlow, NightVentOn, cFanTypes, \&

~~~~~~~~~ FanType\_SimpleConstVolume, FanType\_SimpleVAV, FanType\_SimpleOnOff, FanType\_ZoneExhaust

USE DataGlobals,~~~~ ONLY: SetupOutputVariable, BeginEnvrnFlag, BeginDayFlag, MaxNameLength, \&

~~~~~~ ShowWarningError, ShowFatalError, ShowSevereError, HourofDay, SysSizingCalc, CurrentTime, \&

~~~~~~ OutputFileDebug, ShowContinueError, ShowRecurringWarningErrorAtEnd, WarmupFlag, \&

~ ~~~~~~~~~~~~~~~~~~~~~~~~~ShowContinueErrorTimeStamp

Use DataEnvironment, ONLY: StdBaroPress, DayofMonth, Month, StdRhoAir

USE Psychrometrics,~ ONLY:PsyRhoAirFnPbTdbW, PsyTdbFnHW, PsyCpAirFnW

~ ! Use statements for access to subroutines in other modules

USE ScheduleManager

IMPLICIT NONE~~~~~~~~ ! Enforce explicit typing of all variables

PRIVATE ! Everything private unless explicitly made public

~ !MODULE PARAMETER DEFINITIONS

~ !na

~ ! DERIVED TYPE DEFINITIONS

TYPE FanEquipConditions

~ CHARACTER(len = MaxNameLength) :: FanName~ = `'~ ! Name of the fan

~ CHARACTER(len = MaxNameLength) :: FanType~ = `'~ ! Type of Fan ie. Simple, Vane axial, Centrifugal, etc.

~ CHARACTER(len = MaxNameLength) :: Schedule = `'~ ! Fan Operation Schedule

~ INTEGER ~~~~~:: FanType\_Num~~~~~~~~~~~~~ = 0~~~ ! DataHVACGlobals fan type

~ Integer~~~~~ :: SchedPtr~~~~~~~~~~~~~~~~ = 0~~~ ! Pointer to the correct schedule

~ REAL(r64)~~~ :: InletAirMassFlowRate~~~~ = 0.0~ !MassFlow through the Fan being Simulated {[}kg/Sec{]}

~ REAL(r64)~~~ :: OutletAirMassFlowRate~~~ = 0.0

~ REAL(r64)~~~ :: MaxAirFlowRate~~~~~~~~~~ = 0.0~ !Max Specified Volume Flow Rate of Fan {[}m3/sec{]}

~ REAL(r64)~~~ :: MinAirFlowRate~~~~~~~~~~ = 0.0~ !Min Specified Volume Flow Rate of Fan {[}m3/sec{]}

~ REAL(r64)~~~ :: MaxAirMassFlowRate~~~~~~ = 0.0~ ! Max flow rate of fan in kg/sec

~ REAL(r64)~~~ :: MinAirMassFlowRate~~~~~~ = 0.0~ ! Min flow rate of fan in kg/sec

~ REAL(r64)~~~ :: InletAirTemp~~~~~~~~~~~~ = 0.0

~ REAL(r64)~~~ :: OutletAirTemp~~~~~~~~~~~ = 0.0

~ REAL(r64)~~~ :: InletAirHumRat~~~~~~~~~~ = 0.0

~ REAL(r64)~~~ :: OutletAirHumRat~~~~~~~~~ = 0.0

~ REAL(r64)~~~ :: InletAirEnthalpy~~~~~~~~ = 0.0

~ REAL(r64)~~~ :: OutletAirEnthalpy~~~~~~~ = 0.0

~ REAL(r64)~~~ :: FanPower~~~~~~~~~~~~~~~~ = 0.0~ !Power of the Fan being Simulated {[}kW{]}

~ REAL(r64)~~~ :: FanEnergy~~~~~~~~~~~~~~~ = 0.0~ !Fan energy in {[}kJ{]}

~ REAL(r64)~~~ :: FanRuntimeFraction~~~~~~ = 0.0~ !Fraction of the timestep that the fan operates

~ REAL(r64)~~~ :: DeltaTemp~~~~~~~~~~~~~~~ = 0.0~ !Temp Rise across the Fan {[}C{]}

~ REAL(r64)~~~ :: DeltaPress~~~~~~~~~~~~~~ = 0.0~ !Delta Pressure Across the Fan {[}N/m2{]}

~ REAL(r64)~~~ :: FanEff~~~~~~~~~~~~~~~~~~ = 0.0~ !Fan total efficiency; motor and mechanical

~ REAL(r64)~~~ :: MotEff~~~~~~~~~~~~~~~~~~ = 0.0~ !Fan motor efficiency

~ REAL(r64)~~~ :: MotInAirFrac~~~~~~~~~~~~ = 0.0~ !Fraction of motor heat entering air stream

~ REAL(r64), Dimension(5):: FanCoeff~~~~~~~~~~~ = 0.0~ !Fan Part Load Coefficients to match fan type

~ ! Mass Flow Rate Control Variables

~ REAL(r64)~~~ :: MassFlowRateMaxAvail~~~~ = 0.0

~ REAL(r64)~~~ :: MassFlowRateMinAvail~~~~ = 0.0

~ REAL(r64)~~~ :: RhoAirStdInit~~~~~~~~~~~ = 0.0

~ INTEGER~~~~~ :: InletNodeNum~~~~~~~~~~~~ = 0

~ INTEGER~~~~~ :: OutletNodeNum~~~~~~~~~~~ = 0

~ INTEGER~~~~~ :: NVPerfNum~~~~~~~~~~~~~~~ = 0

~ INTEGER~~~~~ :: FanPowerRatAtSpeedRatCurveIndex~ = 0

~ INTEGER~~~~~ :: FanEffRatioCurveIndex~~~ = 0

~ CHARACTER(len = MaxNameLength) :: EndUseSubcategoryName = `'

~ LOGICAL~~~~~ :: OneTimePowerRatioCheck = .TRUE. ! one time flag used for error message

~ LOGICAL~~~~~ :: OneTimeEffRatioCheck = .TRUE.~~ ! one time flag used for error message

END TYPE FanEquipConditions

TYPE NightVentPerfData

~ CHARACTER(len = MaxNameLength) :: FanName~ = `' ! Name of the fan that will use this data

~ REAL(r64)~~~ :: FanEff~~~~~~~~~~~~~~~~~~ = 0.0 !Fan total efficiency; motor and mechanical

~ REAL(r64)~~~ :: DeltaPress~~~~~~~~~~~~~~ = 0.0 !Delta Pressure Across the Fan {[}N/m2{]}

~ REAL(r64)~~~ :: MaxAirFlowRate~~~~~~~~~~ = 0.0 !Max Specified Volume Flow Rate of Fan {[}m3/s{]}

~ REAL(r64)~~~ :: MaxAirMassFlowRate~~~~~~ = 0.0 ! Max flow rate of fan in kg/sec

~ REAL(r64)~~~ :: MotEff~~~~~~~~~~~~~~~~~~ = 0.0 !Fan motor efficiency

~ REAL(r64)~~~ :: MotInAirFrac~~~~~~~~~~~~ = 0.0 !Fraction of motor heat entering air stream

END TYPE NightVentPerfData

~ !MODULE VARIABLE DECLARATIONS:

~ INTEGER :: NumFans~~~~ = 0 ! The Number of Fans found in the Input

~ INTEGER :: NumNightVentPerf = 0 ! number of FAN:NIGHT VENT PERFORMANCE objects found in the input

~ TYPE (FanEquipConditions), ALLOCATABLE, DIMENSION(:) :: Fan

~ TYPE (NightVentPerfData), ALLOCATABLE, DIMENSION(:)~ :: NightVentPerf

~ LOGICAL :: GetFanInputFlag = .True.~ ! Flag set to make sure you get input once

! Subroutine Specifications for the Module

~~~~~~~~~ ! Driver/Manager Routines

\textbf{Public~ SimulateFanComponents}

~~~~~~~~~ ! Get Input routines for module

PRIVATE GetFanInput

~~~~~~~~~ ! Initialization routines for module

PRIVATE InitFan

PRIVATE SizeFan

~~~~~~~~~ ! Algorithms for the module

Private SimSimpleFan

PRIVATE SimVariableVolumeFan

PRIVATE SimZoneExhaustFan

~~~~~~~~~ ! Update routine to check convergence and update nodes

Private UpdateFan

~~~~~~~~~ ! Reporting routines for module

Private ReportFan

CONTAINS

! MODULE SUBROUTINES:

!*************************************************************************

\textbf{SUBROUTINE SimulateFanComponents(CompName,FirstHVACIteration)}

~~~~~~~~~ ! SUBROUTINE INFORMATION:

~~~~~~~~~ !~~~~~~ AUTHOR~~~~~~~~ Richard Liesen

~~~~~~~~~ !~~~~~~ DATE WRITTEN~~ February 1998

~~~~~~~~~ !~~~~~~ MODIFIED~~~~~~ na

~~~~~~~~~ !~~~~~~ RE-ENGINEERED~ na

~~~~~~~~~ ! PURPOSE OF THIS SUBROUTINE:

~~~~~~~~~ ! This subroutine manages Fan component simulation.

~~~~~~~~~ ! METHODOLOGY EMPLOYED:

~~~~~~~~~ ! na

~~~~~~~~~ ! REFERENCES:

~~~~~~~~~ ! na

~~~~~~~~~ ! USE STATEMENTS:

~ USE InputProcessor, ONLY: FindItemInList

~ IMPLICIT NONE~~~ ! Enforce explicit typing of all variables in this routine

~~~~~~~~~ ! SUBROUTINE ARGUMENT DEFINITIONS:

~ CHARACTER(len = *), INTENT(IN) :: CompName

~ LOGICAL,~~~~~ INTENT (IN):: FirstHVACIteration

~~~~~~~~~ ! SUBROUTINE PARAMETER DEFINITIONS:

~~~~~~~~~ ! na

~~~~~~~~~ ! INTERFACE BLOCK SPECIFICATIONS

~~~~~~~~~ ! DERIVED TYPE DEFINITIONS

~~~~~~~~~ ! na

~~~~~~~~~ ! SUBROUTINE LOCAL VARIABLE DECLARATIONS:

~ INTEGER~~~~~~~~~~~~ :: FanNum~~~~ ! current fan number

~ LOGICAL,SAVE~~~~~~~ :: GetInputFlag = .True.~ ! Flag set to make sure you get input once

~~~~~~~~~ ! FLOW:

~ ! Obtains and Allocates fan related parameters from input file

~ IF (GetInputFlag) THEN~ !First time subroutine has been entered

~~~ CALL GetFanInput

~~~ GetInputFlag = .false.

~ End If

~ ! Find the correct FanNumber with the AirLoop \& CompNum from AirLoop Derived Type

~ !FanNum = AirLoopEquip(AirLoopNum)\%ComponentOfTypeNum(CompNum)

~ ! Determine which Fan given the Fan Name

~ FanNum = ~~ FindItemInList(CompName,Fan\%FanName,NumFans)

~ IF (FanNum = = 0) THEN

~~~ CALL ShowFatalError(`Fan not found ='//TRIM(CompName))

~ ENDIF

~ ! With the correct FanNum Initialize

~ CALL InitFan(FanNum,FirstHVACIteration)~ ! Initialize all fan related parameters

~ ! Calculate the Correct Fan Model with the current FanNum

~ IF (Fan(FanNum)\%FanType\_Num = = FanType\_SimpleConstVolume) THEN

~~~ Call SimSimpleFan(FanNum)

~ Else IF (Fan(FanNum)\%FanType\_Num = = FanType\_SimpleVAV) THEN

~~~ Call SimVariableVolumeFan(FanNum)

~ Else If (Fan(FanNum)\%FanType\_Num = = FanType\_SimpleOnOff) THEN

~~~ Call SimOnOffFan(FanNum)

~ Else If (Fan(FanNum)\%FanType\_Num = = FanType\_ZoneExhaust) THEN

~~~ Call SimZoneExhaustFan(FanNum)

~ End If

~ ! Update the current fan to the outlet nodes

~ Call UpdateFan(FanNum)

~ ! Report the current fan

~ Call ReportFan(FanNum)

~ RETURN

\textbf{END SUBROUTINE SimulateFanComponents}

! Get Input Section of the Module

!******************************************************************************

\textbf{SUBROUTINE GetFanInput}

~~~~~~~~~ ! SUBROUTINE INFORMATION:

~~~~~~~~~ !~~~~~~ AUTHOR~~~~~~~~ Richard Liesen

~~~~~~~~~ !~~~~~~ DATE WRITTEN~~ April 1998

~~~~~~~~~ !~~~~~~ MODIFIED~~~~~~ Shirey, May 2001

~~~~~~~~~ !~~~~~~ RE-ENGINEERED~ na

~~~~~~~~~ ! PURPOSE OF THIS SUBROUTINE:

~~~~~~~~~ ! Obtains input data for fans and stores it in fan data structures

~~~~~~~~~ ! METHODOLOGY EMPLOYED:

~~~~~~~~~ ! Uses ``Get'' routines to read in data.

~~~~~~~~~ ! REFERENCES:

~~~~~~~~~ ! na

~~~~~~~~~ ! USE STATEMENTS:

~~~ USE InputProcessor

~~~ USE NodeInputManager,~~~~~ ONLY: GetOnlySingleNode

~~~ USE CurveManager,~~~~~~~~~ ONLY: GetCurveIndex

~~~ USE BranchNodeConnections, ONLY: TestCompSet

!~~~ USE DataIPShortCuts

~ IMPLICIT NONE~~~ ! Enforce explicit typing of all variables in this routine

~~~~~~~~~ ! SUBROUTINE ARGUMENT DEFINITIONS:

~~~~~~~~~ ! na

~~~~~~~~~ ! SUBROUTINE PARAMETER DEFINITIONS:

~~~~~~~~~ ! na

~~~~~~~~~ ! INTERFACE BLOCK SPECIFICATIONS

~~~~~~~~~ ! na

~~~~~~~~~ ! DERIVED TYPE DEFINITIONS

~~~~~~~~~ ! na

~~~~~~~~~ ! SUBROUTINE LOCAL VARIABLE DECLARATIONS:

~~~ INTEGER :: FanNum~~~~~ ! The fan that you are currently loading input into

~~~ INTEGER :: NumSimpFan~ ! The number of Simple Const Vol Fans

~~~ INTEGER :: NumVarVolFan ! The number of Simple Variable Vol Fans

~~~ INTEGER :: NumOnOff~~~~ ! The number of Simple on-off Fans

~~~ INTEGER :: NumZoneExhFan

~~~ INTEGER :: SimpFanNum

~~~ INTEGER :: OnOffFanNum

~~~ INTEGER :: VarVolFanNum

~~~ INTEGER :: ExhFanNum

~~~ INTEGER :: NVPerfNum

~~~ LOGICAL :: NVPerfFanFound

~~~ INTEGER :: NumAlphas

~~~ INTEGER :: NumNums

~~~ INTEGER :: IOSTAT

~~~ LOGICAL :: ErrorsFound = .false.~~ ! If errors detected in input

~~~ LOGICAL :: IsNotOK~~~~~~~~~~~~~~ ! Flag to verify name

~~~ LOGICAL :: IsBlank~~~~~~~~~~~~~~ ! Flag for blank name

~~~ CHARACTER(len = *), PARAMETER~~~ :: RoutineName = `GetFanInput:' ! include trailing blank space

~~~ CHARACTER(len = MaxNameLength+40),ALLOCATABLE, DIMENSION(:) :: cAlphaFieldNames

~~~ CHARACTER(len = MaxNameLength+40),ALLOCATABLE, DIMENSION(:) :: cNumericFieldNames

~~~ LOGICAL, ALLOCATABLE, DIMENSION(:) :: lNumericFieldBlanks

~~~ LOGICAL, ALLOCATABLE, DIMENSION(:) :: lAlphaFieldBlanks

~~~ CHARACTER(len = MaxNameLength),ALLOCATABLE, DIMENSION(:) :: cAlphaArgs

~~~ REAL(r64),ALLOCATABLE, DIMENSION(:) :: rNumericArgs

~~~ CHARACTER(len = MaxNameLength) :: cCurrentModuleObject

~~~ INTEGER :: NumParams

~~~ INTEGER :: MaxAlphas

~~~ INTEGER :: MaxNumbers

~~~~~~~~~ ! Flow

~~~ MaxAlphas = 0

~~~ MaxNumbers = 0

~~~ NumSimpFan~~ = GetNumObjectsFound(`Fan:ConstantVolume')

~~~ IF (NumSimpFan \textgreater{} 0) THEN

~~~~~ CALL GetObjectDefMaxArgs(`Fan:ConstantVolume',NumParams,NumAlphas,NumNums)

~~~~~ MaxAlphas = MAX(MaxAlphas,NumAlphas)

~~~~~ MaxNumbers = MAX(MaxNumbers,NumNums)

~~~ ENDIF

~~~ NumVarVolFan = GetNumObjectsFound(`Fan:VariableVolume')

~~~ IF (NumVarVolFan \textgreater{} 0) THEN

~~~~~ CALL GetObjectDefMaxArgs(`Fan:VariableVolume',NumParams,NumAlphas,NumNums)

~~~~~ MaxAlphas = MAX(MaxAlphas,NumAlphas)

~~~~~ MaxNumbers = MAX(MaxNumbers,NumNums)

~~~ ENDIF

~~~ NumOnOff = GetNumObjectsFound(`Fan:OnOff')

~~~ IF (NumOnOff \textgreater{} 0) THEN

~~~~~ CALL GetObjectDefMaxArgs(`Fan:OnOff',NumParams,NumAlphas,NumNums)

~~~~~ MaxAlphas = MAX(MaxAlphas,NumAlphas)

~~~~~ MaxNumbers = MAX(MaxNumbers,NumNums)

~~~ ENDIF

~~~ NumZoneExhFan = GetNumObjectsFound(`Fan:ZoneExhaust')

~~~ IF (NumZoneExhFan \textgreater{} 0) THEN

~~~~~ CALL GetObjectDefMaxArgs(`Fan:ZoneExhaust',NumParams,NumAlphas,NumNums)

~~~~~ MaxAlphas = MAX(MaxAlphas,NumAlphas)

~~~~~ MaxNumbers = MAX(MaxNumbers,NumNums)

~~~ ENDIF

~~~ NumNightVentPerf = GetNumObjectsFound(`FanPerformance:NightVentilation')

~~~ IF (NumNightVentPerf \textgreater{} 0) THEN

~~~~~ CALL GetObjectDefMaxArgs(`FanPerformance:NightVentilation',NumParams,NumAlphas,NumNums)

~~~~~ MaxAlphas = MAX(MaxAlphas,NumAlphas)

~~~~~ MaxNumbers = MAX(MaxNumbers,NumNums)

~~~ ENDIF

~~~ ALLOCATE(cAlphaArgs(MaxAlphas))

~~~ cAlphaArgs = `'

~~~ ALLOCATE(cAlphaFieldNames(MaxAlphas))

~~~ cAlphaFieldNames = `'

~~~ ALLOCATE(lAlphaFieldBlanks(MaxAlphas))

~~~ lAlphaFieldBlanks = .false.

~~~ ALLOCATE(cNumericFieldNames(MaxNumbers))

~~~ cNumericFieldNames = `'

~~~ ALLOCATE(lNumericFieldBlanks(MaxNumbers))

~~~ lNumericFieldBlanks = .false.

~~~ ALLOCATE(rNumericArgs(MaxNumbers))

~~~ rNumericArgs = 0.0

~~~ NumFans = NumSimpFan + NumVarVolFan + NumZoneExhFan+NumOnOff

~~~ IF (NumFans \textgreater{} 0) THEN

~~~~~ ALLOCATE(Fan(NumFans))

~~~ ENDIF

~~~~~ DO SimpFanNum = 1,~ NumSimpFan

~~~~~~~ FanNum = SimpFanNum

~~~~~~~ cCurrentModuleObject = `Fan:ConstantVolume'

~~~~~~~ CALL GetObjectItem(TRIM(cCurrentModuleObject),SimpFanNum,cAlphaArgs,NumAlphas, \&

~~~~~~~~~~~~~~~~~~~~~~~~~~ rNumericArgs,NumNums,IOSTAT, \&

~~~~~~~~~~~~~~~~~~~~~~~~~~ NumBlank = lNumericFieldBlanks,AlphaBlank = lAlphaFieldBlanks, \&

~~~~~~~~~~~~~~~~~~~~~~~~~~ AlphaFieldNames = cAlphaFieldNames,NumericFieldNames = cNumericFieldNames)

~~~~~~~ IsNotOK = .false.

~~~~~~~ IsBlank = .false.

~~~~~~~ CALL VerifyName(cAlphaArgs(1),Fan\%FanName,FanNum-1,IsNotOK,IsBlank,TRIM(cCurrentModuleObject)//`Name')

~~~~~~~ IF (IsNotOK) THEN

~~~~~~~~~ ErrorsFound = .true.

~~~~~~~~~ IF (IsBlank) cAlphaArgs(1) = `xxxxx'

~~~~~~~ ENDIF

~~~~~~~ Fan(FanNum)\%FanName~ = cAlphaArgs(1)

~~~~~~~ Fan(FanNum)\%FanType = ~ cCurrentModuleObject

~~~~~~~ Fan(FanNum)\%Schedule = cAlphaArgs(2)

~~~~~~~ Fan(FanNum)\%SchedPtr = GetScheduleIndex(cAlphaArgs(2))

~~~~~~~ IF (Fan(FanNum)\%SchedPtr = = 0) THEN

~~~~~~~~~ IF (lAlphaFieldBlanks(2)) THEN

~~~~~~~~~~~ CALL ShowSevereError(RoutineName//TRIM(cCurrentModuleObject)//`:'//TRIM(cAlphaFieldNames(2))//~ \&

~~~~~~~~~~~~~~~~ `is required, missing for'//TRIM(cAlphaFieldNames(1))//`='//TRIM(cAlphaArgs(1)))

~~~~~~~~~ ELSE

~~~~~~~~~~~ CALL ShowSevereError(RoutineName//TRIM(cCurrentModuleObject)//`: invalid'//TRIM(cAlphaFieldNames(2))//~ \&

~~~~~~~~~~~~~~ `entered ='//TRIM(cAlphaArgs(2))// \&

~~~~~~~~~~~~~~ `for'//TRIM(cAlphaFieldNames(1))//`='//TRIM(cAlphaArgs(1)))

~~~~~~~~~ END IF

~~~~~~~~~ ErrorsFound = .true.

~~~~~~~ END IF

!~~~~~~~ Fan(FanNum)\%Control = `CONSTVOLUME'

~~~~~~~ Fan(FanNum)\%FanType\_Num = FanType\_SimpleConstVolume

~~~~~~~ Fan(FanNum)\%FanEff~~~~~~~ = rNumericArgs(1)

~~~~~~~ Fan(FanNum)\%DeltaPress~~~ = rNumericArgs(2)

~~~~~~~ Fan(FanNum)\%MaxAirFlowRate = rNumericArgs(3)

~~~~~~~ IF (Fan(FanNum)\%MaxAirFlowRate = = 0.0) THEN

~~~~~~~~~ CALL ShowWarningError(TRIM(cCurrentModuleObject)//' = ``'//TRIM(Fan(FanNum)\%FanName)//~ \&

~~~~~~~~~~~~ `" has specified 0.0 max air flow rate. It will not be used in the simulation.')

~~~~~~~ ENDIF

~~~~~~~ Fan(FanNum)\%MotEff~~~~~~~ = rNumericArgs(4)

~~~~~~~ Fan(FanNum)\%MotInAirFrac~ = rNumericArgs(5)

~~~~~~~ Fan(FanNum)\%MinAirFlowRate = 0.0

~~~~~~~ Fan(FanNum)\%InletNodeNum~ = \&

~~~~~~~~~~~~~~ GetOnlySingleNode(cAlphaArgs(3),ErrorsFound,TRIM(cCurrentModuleObject),cAlphaArgs(1),~ \&

~~~~~~~~ ~~~~~~~~~~~~~~~~~~~NodeType\_Air,NodeConnectionType\_Inlet,1,ObjectIsNotParent)

~~~~~~~ Fan(FanNum)\%OutletNodeNum = \&

~~~~~~~~~~~~~~ GetOnlySingleNode(cAlphaArgs(4),ErrorsFound,TRIM(cCurrentModuleObject),cAlphaArgs(1),~ \&

~~~~~~~~~~~~~~~~~~~~~~~~~~~ NodeType\_Air,NodeConnectionType\_Outlet,1,ObjectIsNotParent)

~~~~~~~ IF (NumAlphas \textgreater{} 4) THEN

~~~~~~~~~ Fan(FanNum)\%EndUseSubcategoryName = cAlphaArgs(5)

~~~~~~~ ELSE

~~~~~~~~~ Fan(FanNum)\%EndUseSubcategoryName = `General'

~~~~~~~ END IF

~~~~~~~ CALL TestCompSet(TRIM(cCurrentModuleObject),cAlphaArgs(1),cAlphaArgs(3),cAlphaArgs(4),`Air Nodes')

~~~~~ END DO~~ ! end Number of Simple FAN Loop

~~~~~ DO VarVolFanNum = 1,~ NumVarVolFan

~~~~~~~ FanNum = NumSimpFan + VarVolFanNum

~~~~~~~ cCurrentModuleObject = `Fan:VariableVolume'

~~~~~~~ CALL GetObjectItem(TRIM(cCurrentModuleObject),VarVolFanNum,cAlphaArgs,NumAlphas, \&

~~~~~~~~~~~~~~~~~~~~~~~~~~ rNumericArgs,NumNums,IOSTAT, \&

~~~~~~~~~~~~~~~~~~~~~~~~~~ NumBlank = lNumericFieldBlanks,AlphaBlank = lAlphaFieldBlanks, \&

~~~~~~~~~~~~~~~~~~~~~~~~~~ AlphaFieldNames = cAlphaFieldNames,NumericFieldNames = cNumericFieldNames)

~~~~~~~ IsNotOK = .false.

~~~~~~~ IsBlank = .false.

~~~~~~~ CALL VerifyName(cAlphaArgs(1),Fan\%FanName,FanNum-1,IsNotOK,IsBlank,TRIM(cCurrentModuleObject)//`Name')

~~~~~~~ IF (IsNotOK) THEN

~~~~~~~~~ ErrorsFound = .true.

~~~~~~~~~ IF (IsBlank) cAlphaArgs(1) = `xxxxx'

~~~~~~~ ENDIF

~~~~~~~ Fan(FanNum)\%FanName = cAlphaArgs(1)

~~~~~~~ Fan(FanNum)\%FanType = cCurrentModuleObject

~~~~~~~ Fan(FanNum)\%Schedule = cAlphaArgs(2)

~~~~~~~ Fan(FanNum)\%SchedPtr = GetScheduleIndex(cAlphaArgs(2))

~~~~~~~ IF (Fan(FanNum)\%SchedPtr = = 0) THEN

~~~~~~~~~ IF (lAlphaFieldBlanks(2)) THEN

~~~~~~~~~~~ CALL ShowSevereError(RoutineName//TRIM(cCurrentModuleObject)//`:'//TRIM(cAlphaFieldNames(2))//~ \&

~~~~~~~~~~~~~~~~ `is required, missing for'//TRIM(cAlphaFieldNames(1))//`='//TRIM(cAlphaArgs(1)))

~~~~~~~~~ ELSE

~~~~~~~~~~~ CALL ShowSevereError(RoutineName//TRIM(cCurrentModuleObject)//`: invalid'//TRIM(cAlphaFieldNames(2))//~ \&

~~~~~~~~~~~~~~ `entered ='//TRIM(cAlphaArgs(2))// \&

~~~~~~~~~~~~~~ `for'//TRIM(cAlphaFieldNames(1))//`='//TRIM(cAlphaArgs(1)))

~~~~~~~~~ END IF

~~~~~~~~~ ErrorsFound = .true.

~~~~~~~ ENDIF

!~~~~~~~ Fan(FanNum)\%Control = `VARIABLEVOLUME'

~~~~~~~ Fan(FanNum)\%FanType\_Num = FanType\_SimpleVAV

~~~~~~~ Fan(FanNum)\%FanEff~~~~~~~ = rNumericArgs(1)

~~~~~~~ Fan(FanNum)\%DeltaPress~~~ = rNumericArgs(2)

~~~~~~~ Fan(FanNum)\%MaxAirFlowRate = rNumericArgs(3)

~~~~~~~ IF (Fan(FanNum)\%MaxAirFlowRate = = 0.0) THEN

~~~~~~~~~ CALL ShowWarningError(TRIM(cCurrentModuleObject)//' = ``'//TRIM(Fan(FanNum)\%FanName)//~ \&

~~~~~~~~~~~~ `" has specified 0.0 max air flow rate. It will not be used in the simulation.')

~~~~~~~ ENDIF

~~~~~~~ Fan(FanNum)\%MinAirFlowRate = rNumericArgs(4)

~~~~~~~ Fan(FanNum)\%MotEff~~~~~~~ = rNumericArgs(5)

~~~~~~~ Fan(FanNum)\%MotInAirFrac~ = rNumericArgs(6)

~~~~~~~ Fan(FanNum)\%FanCoeff(1)~~ = rNumericArgs(7)

~~~~~~~ Fan(FanNum)\%FanCoeff(2)~~ = rNumericArgs(8)

~~~~~~~ Fan(FanNum)\%FanCoeff(3)~~ = rNumericArgs(9)

~~~~~~~ Fan(FanNum)\%FanCoeff(4)~~ = rNumericArgs(10)

~~~~~~~ Fan(FanNum)\%FanCoeff(5)~~ = rNumericArgs(11)

~~~~~~~ IF (Fan(FanNum)\%FanCoeff(1) = = 0.0 .and. Fan(FanNum)\%FanCoeff(2) = = 0.0 .and.~ \&

~~~~~~~~~~~ Fan(FanNum)\%FanCoeff(3) = = 0.0 .and. Fan(FanNum)\%FanCoeff(4) = = 0.0 .and.~ \&

~~~~~~~~~~~ Fan(FanNum)\%FanCoeff(5) = = 0.0)~ THEN

~~~~~~~~~~~ CALL ShowWarningError(`Fan Coefficients are all zero.~ No Fan power will be reported.')

~~~~~~~~~~~ CALL ShowContinueError(`For'//TRIM(cCurrentModuleObject)//`, Fan ='//TRIM(cAlphaArgs(1)))

~~~~~~~ ENDIF

~~~~~~~ Fan(FanNum)\%InletNodeNum~ = \&

~~~~~~~~~~~~~~ GetOnlySingleNode(cAlphaArgs(3),ErrorsFound,TRIM(cCurrentModuleObject),cAlphaArgs(1),~ \&

~~~~~~~~~~~~~~~~~~~~~~~~~~~ NodeType\_Air,NodeConnectionType\_Inlet,1,ObjectIsNotParent)

~~~~~~~ Fan(FanNum)\%OutletNodeNum = \&

~~~~~~~~~~~~~~ GetOnlySingleNode(cAlphaArgs(4),ErrorsFound,TRIM(cCurrentModuleObject),cAlphaArgs(1),~ \&

~~~~~~~~~~~~~~~~~~~~~~~~~~~ NodeType\_Air,NodeConnectionType\_Outlet,1,ObjectIsNotParent)

~~~~~~~ IF (NumAlphas \textgreater{} 4) THEN

~~~~~~~~~ Fan(FanNum)\%EndUseSubcategoryName = cAlphaArgs(5)

~~~~~~~ ELSE

~~~~~~~~~ Fan(FanNum)\%EndUseSubcategoryName = `General'

~~~~~~~ END IF

~~~~~~~ CALL TestCompSet(TRIM(cCurrentModuleObject),cAlphaArgs(1),cAlphaArgs(3),cAlphaArgs(4),`Air Nodes')

~~~~~ END DO~~ ! end Number of Variable Volume FAN Loop

~~~~~ DO ExhFanNum = 1,~ NumZoneExhFan

~~~~~~~ FanNum = NumSimpFan + NumVarVolFan + ExhFanNum

~~~~~~~ cCurrentModuleObject = `Fan:ZoneExhaust'

~~~~~~~ CALL GetObjectItem(TRIM(cCurrentModuleObject),ExhFanNum,cAlphaArgs,NumAlphas, \&

~~~~~~~~~~~~~~~~~~~~~~~~~~ rNumericArgs,NumNums,IOSTAT, \&

~~~~~~~~~~~~~~~~~~~~~~~~~~ NumBlank = lNumericFieldBlanks,AlphaBlank = lAlphaFieldBlanks, \&

~~~~~~~~~~~~~~~~~~~~~~~~~~ AlphaFieldNames = cAlphaFieldNames,NumericFieldNames = cNumericFieldNames)

~~~~~~~ IsNotOK = .false.

~~~~~~~ IsBlank = .false.

~~~~~~~ CALL VerifyName(cAlphaArgs(1),Fan\%FanName,FanNum-1,IsNotOK,IsBlank,TRIM(cCurrentModuleObject)//`Name')

~~~~~~~ IF (IsNotOK) THEN

~~~~~~~~~ ErrorsFound = .true.

~~~~~~~~~ IF (IsBlank) cAlphaArgs(1) = `xxxxx'

~~~~~~~ ENDIF

~~~~~~~ Fan(FanNum)\%FanName = cAlphaArgs(1)

~~~~~~~ Fan(FanNum)\%FanType = cCurrentModuleObject

~~~~~~~ Fan(FanNum)\%Schedule = cAlphaArgs(2)

~~~~~~~ Fan(FanNum)\%SchedPtr = GetScheduleIndex(cAlphaArgs(2))

~~~~~~~ IF (Fan(FanNum)\%SchedPtr = = 0) THEN

~~~~~~~~~ IF (lAlphaFieldBlanks(2)) THEN

~~~~~~~~~~~ CALL ShowSevereError(RoutineName//TRIM(cCurrentModuleObject)//`:'//TRIM(cAlphaFieldNames(2))//~ \&

~~~~~~~~~~~~~~~~ `is required, missing for'//TRIM(cAlphaFieldNames(1))//`='//TRIM(cAlphaArgs(1)))

~~~~~~~~~ ELSE

~~~~~~~~~~~ CALL ShowSevereError(RoutineName//TRIM(cCurrentModuleObject)//`: invalid'//TRIM(cAlphaFieldNames(2))//~ \&

~~~~~~~~~~~~~~ `entered ='//TRIM(cAlphaArgs(2))// \&

~~~~~~~~~~~~~~ `for'//TRIM(cAlphaFieldNames(1))//`='//TRIM(cAlphaArgs(1)))

~~~~~~~~~ END IF

~~~~~~~~~ ErrorsFound = .true.

~~~~~~~ ELSE

~~~~~~~~~ IF (HasFractionalScheduleValue(Fan(FanNum)\%SchedPtr)) THEN

~~~~~~~~~~~ CALL ShowWarningError(TRIM(cCurrentModuleObject)//' = ``'//TRIM(Fan(FanNum)\%FanName)//~ \&

~~~~~~~~~~~~~ `" has fractional values in Schedule ='//TRIM(cAlphaArgs(2))//`. Only 0.0 in the schedule value turns the fan off.')

~~~~~~~~~ ENDIF

~~~~~~~ ENDIF

!~~~~~~~ Fan(FanNum)\%Control = `CONSTVOLUME'

~~~~~~~ Fan(FanNum)\%FanType\_Num = FanType\_ZoneExhaust

~~~~~~~ Fan(FanNum)\%FanEff~~~~~~~ = rNumericArgs(1)

~~~~~~~ Fan(FanNum)\%DeltaPress~~~ = rNumericArgs(2)

~~~~~~~ Fan(FanNum)\%MaxAirFlowRate = rNumericArgs(3)

~~~~~~~ Fan(FanNum)\%MotEff~~~~~~~ = 1.0

~~~~~~~ Fan(FanNum)\%MotInAirFrac~ = 1.0

~~~~~~~ Fan(FanNum)\%MinAirFlowRate = 0.0

~~~~~~~ Fan(FanNum)\%RhoAirStdInit = StdRhoAir

~~~~~~~ Fan(FanNum)\%MaxAirMassFlowRate = Fan(FanNum)\%MaxAirFlowRate * Fan(FanNum)\%RhoAirStdInit

~~~~~~~ IF (Fan(FanNum)\%MaxAirFlowRate = = 0.0) THEN

~~~~~~~~~ CALL ShowWarningError(TRIM(cCurrentModuleObject)//' = ``'//TRIM(Fan(FanNum)\%FanName)//~ \&

~~~~~~~~~~~~~ `" has specified 0.0 max air flow rate. It will not be used in the simulation.')

~~~~~~~ ENDIF

~~~~~~~ Fan(FanNum)\%InletNodeNum~ = \&

~~~~~~~~~~~~~~ GetOnlySingleNode(cAlphaArgs(3),ErrorsFound,TRIM(cCurrentModuleObject),cAlphaArgs(1),~ \&

~~~~~~~~~~~~~~~~~~~~~~~~~~~ NodeType\_Air,NodeConnectionType\_Inlet,1,ObjectIsNotParent)

~~~~~~~ Fan(FanNum)\%OutletNodeNum = \&

~~~~~~~~~~~~~~ GetOnlySingleNode(cAlphaArgs(4),ErrorsFound,TRIM(cCurrentModuleObject),cAlphaArgs(1),~ \&

~~~~~~~~~~~~~~~~~~~~~~~~~~~ NodeType\_Air,NodeConnectionType\_Outlet,1,ObjectIsNotParent)

~~~~~~~ IF (NumAlphas \textgreater{} 4) THEN

~~~~~~~~~ Fan(FanNum)\%EndUseSubcategoryName = cAlphaArgs(5)

~~~~~~~ ELSE

~~~~~~~~~ Fan(FanNum)\%EndUseSubcategoryName = `General'

~~~~~~~ END IF

~~~~~~~ ! Component sets not setup yet for zone equipment

~~~~~~~ ! CALL TestCompSet(TRIM(cCurrentModuleObject),cAlphaArgs(1),cAlphaArgs(3),cAlphaArgs(4),`Air Nodes')

~~~~~ END DO~~ ! end of Zone Exhaust Fan loop

~~~~~ DO OnOffFanNum = 1,~ NumOnOff

~~~~~~~ FanNum = NumSimpFan + NumVarVolFan + NumZoneExhFan + OnOffFanNum

~~~~~~~ cCurrentModuleObject = `Fan:OnOff'

~~~~~~~ CALL GetObjectItem(TRIM(cCurrentModuleObject),OnOffFanNum,cAlphaArgs,NumAlphas, \&

~~~~~~~~~~~~~~~~~~~~~~~~~~ rNumericArgs,NumNums,IOSTAT, \&

~~~~~~~~~~~~~~~~~~~~~~~~~~ NumBlank = lNumericFieldBlanks,AlphaBlank = lAlphaFieldBlanks, \&

~~~~~~~~~~~~~~~~~~~ ~~~~~~~AlphaFieldNames = cAlphaFieldNames,NumericFieldNames = cNumericFieldNames)

~~~~~~~ IsNotOK = .false.

~~~~~~~ IsBlank = .false.

~~~~~~~ CALL VerifyName(cAlphaArgs(1),Fan\%FanName,FanNum-1,IsNotOK,IsBlank,TRIM(cCurrentModuleObject)//`Name')

~~~~~~~ IF (IsNotOK) THEN

~~~~~~~~~ ErrorsFound = .true.

~~~~~~~~~ IF (IsBlank) cAlphaArgs(1) = `xxxxx'

~~~~~~~ ENDIF

~~~~~~~ Fan(FanNum)\%FanName~ = cAlphaArgs(1)

~~~~~~~ Fan(FanNum)\%FanType~ = cCurrentModuleObject

~~~~~~~ Fan(FanNum)\%Schedule = cAlphaArgs(2)

~~~~~~~ Fan(FanNum)\%SchedPtr = GetScheduleIndex(cAlphaArgs(2))

~~~~~~~ IF (Fan(FanNum)\%SchedPtr = = 0) THEN

~~~~~~~~~ IF (lAlphaFieldBlanks(2)) THEN

~~~~~~~~~~~ CALL ShowSevereError(RoutineName//TRIM(cCurrentModuleObject)//`:'//TRIM(cAlphaFieldNames(2))//~ \&

~~~~~~~~~~~~~~ ~~`is required, missing for'//TRIM(cAlphaFieldNames(1))//`='//TRIM(cAlphaArgs(1)))

~~~~~~~~~ ELSE

~~~~~~~~~~~ CALL ShowSevereError(RoutineName//TRIM(cCurrentModuleObject)//`: invalid'//TRIM(cAlphaFieldNames(2))//~ \&

~~~~~~~~~~~~~~ `entered ='//TRIM(cAlphaArgs(2))// \&

~~~~~~~~~~~~~~ `for'//TRIM(cAlphaFieldNames(1))//`='//TRIM(cAlphaArgs(1)))

~~~~~~~~~ END IF

~~~~~~~~~ ErrorsFound = .true.

~~~~~~~ ENDIF

!~~~~~~~ Fan(FanNum)\%Control = `ONOFF'

~~~~~~~ Fan(FanNum)\%FanType\_Num = FanType\_SimpleOnOff

~~~~~~~ Fan(FanNum)\%FanEff~~~~~~~ = rNumericArgs(1)

~~~~~~~ Fan(FanNum)\%DeltaPress~~~ = rNumericArgs(2)

~~~~~~~ Fan(FanNum)\%MaxAirFlowRate = rNumericArgs(3)

~~~~~~~ IF (Fan(FanNum)\%MaxAirFlowRate = = 0.0) THEN

~~~~~~~~~ CALL ShowWarningError(TRIM(cCurrentModuleObject)//' = ``'//TRIM(Fan(FanNum)\%FanName)//~ \&

~~~~~~~~~~~~~ `" has specified 0.0 max air flow rate. It will not be used in the simulation.')

~~~~~~~ ENDIF

!~~~~~~ the following two structure variables are set here, as well as in InitFan, for the Heat Pump:Water Heater object

!~~~~~~ (Standard Rating procedure may be called before BeginEnvirFlag is set to TRUE, if so MaxAirMassFlowRate = 0)

~~~~~~~ Fan(FanNum)\%RhoAirStdInit = StdRhoAir

~~~~~~~ Fan(FanNum)\%MaxAirMassFlowRate = Fan(FanNum)\%MaxAirFlowRate * Fan(FanNum)\%RhoAirStdInit

~~~~~~~ Fan(FanNum)\%MotEff~~~~~~~ = rNumericArgs(4)

~~~~~~~ Fan(FanNum)\%MotInAirFrac~ = rNumericArgs(5)

~~~~~~~ Fan(FanNum)\%MinAirFlowRate = 0.0

~~~~~~~ Fan(FanNum)\%InletNodeNum~ = \&

~~~~~~~~~~~~~~ GetOnlySingleNode(cAlphaArgs(3),ErrorsFound,TRIM(cCurrentModuleObject),cAlphaArgs(1), \&

~~~~~~~~~~~~~~~~~~~~~~~~~~~ NodeType\_Air,NodeConnectionType\_Inlet,1,ObjectIsNotParent)

~~~~~~~ Fan(FanNum)\%OutletNodeNum = \&

~~~~~~~~~~~~~~ GetOnlySingleNode(cAlphaArgs(4),ErrorsFound,TRIM(cCurrentModuleObject),cAlphaArgs(1), \&

~~~~~~~~~~~~~~~~~~~~~~~~~~~ NodeType\_Air,NodeConnectionType\_Outlet,1,ObjectIsNotParent)

~~~~~~~ IF (NumAlphas \textgreater{} 4 .AND. .NOT. lAlphaFieldBlanks(5)) THEN

~~~~~~~~~ Fan(FanNum)\%FanPowerRatAtSpeedRatCurveIndex~ = GetCurveIndex(cAlphaArgs(5))

~~~~~~~ END IF

~~~~~~~ IF (NumAlphas \textgreater{} 5 .AND. .NOT. lAlphaFieldBlanks(6)) THEN

~~~~~~~~~ Fan(FanNum)\%FanEffRatioCurveIndex~ = GetCurveIndex(cAlphaArgs(6))

~~~~~~~ END IF

~~~~~~~ IF (NumAlphas \textgreater{} 6 .AND. .NOT. lAlphaFieldBlanks(7)) THEN

~~~~~~~~~ Fan(FanNum)\%EndUseSubcategoryName = cAlphaArgs(7)

~~~~~~~ ELSE

~~~~~~~~~ Fan(FanNum)\%EndUseSubcategoryName = `General'

~~~~~~~ END IF

~~~~~~~ CALL TestCompSet(TRIM(cCurrentModuleObject),cAlphaArgs(1),cAlphaArgs(3),cAlphaArgs(4),`Air Nodes')

~ ~~~~END DO~~ ! end Number of Simple~ ON-OFF FAN Loop

~~~~~ cCurrentModuleObject = `FanPerformance:NightVentilation'

~~~~~ NumNightVentPerf = GetNumObjectsFound(TRIM(cCurrentModuleObject))

~~~~~ IF (NumNightVentPerf \textgreater{} 0) THEN

~~~~~~~ ALLOCATE(NightVentPerf(NumNightVentPerf))

~~~~~~~ NightVentPerf\%FanName = `'

~~~~~~~ NightVentPerf\%FanEff = 0.0

~~~~~~~ NightVentPerf\%DeltaPress = 0.0

~~~~~~~ NightVentPerf\%MaxAirFlowRate = 0.0

~~~~~~~ NightVentPerf\%MotEff = 0.0

~~~~~~~ NightVentPerf\%MotInAirFrac = 0.0

~~~~~~~ NightVentPerf\%MaxAirMassFlowRate = 0.0

~~~~~ END IF

~~~~~ ! input the night ventilation performance objects

~~~~~ DO NVPerfNum = 1,NumNightVentPerf

~~~~~~~~ CALL GetObjectItem(TRIM(cCurrentModuleObject),NVPerfNum,cAlphaArgs,NumAlphas, \&

~~~~~~~~~~~~ ~~~~~~~~~~~~~~rNumericArgs,NumNums,IOSTAT, \&

~~~~~~~~~~~~~~~~~~~~~~~~~~ NumBlank = lNumericFieldBlanks,AlphaBlank = lAlphaFieldBlanks, \&

~~~~~~~~~~~~~~~~~~~~~~~~~~ AlphaFieldNames = cAlphaFieldNames,NumericFieldNames = cNumericFieldNames)

~~~~~~~ IsNotOK = .false.

~~~~~~~ IsBlank = .false.

~~~~~~~ CALL VerifyName(cAlphaArgs(1),NightVentPerf\%FanName,NVPerfNum-1,IsNotOK,IsBlank,TRIM(cCurrentModuleObject)//`Name')

~~~~~~~ IF (IsNotOK) THEN

~~~~~~~~~ ErrorsFound = .true.

~~~~~~~~~ IF (IsBlank) cAlphaArgs(1) = `xxxxx'

~~~~~~~ ENDIF

~~~~~~~ NightVentPerf(NVPerfNum)\%FanName~~~~~~~ = cAlphaArgs(1)

~~~~~~~ NightVentPerf(NVPerfNum)\%FanEff~~~~~~~~ = rNumericArgs(1)

~~~~~~~ NightVentPerf(NVPerfNum)\%DeltaPress~~~~ = rNumericArgs(2)

~~~~~~~ NightVentPerf(NVPerfNum)\%MaxAirFlowRate = rNumericArgs(3)

~~~~~~~ NightVentPerf(NVPerfNum)\%MotEff~~~~~~~~ = rNumericArgs(4)

~~~~~~~ NightVentPerf(NVPerfNum)\%MotInAirFrac~~ = rNumericArgs(5)

~~~~~~~ ! find the corresponding fan

~~~~~~~ NVPerfFanFound = .FALSE.

~~~~~~~ DO FanNum = 1,NumFans

~~~~~~~~~ IF (NightVentPerf(NVPerfNum)\%FanName = = Fan(FanNum)\%FanName) THEN

~~~~~~~~~~~ NVPerfFanFound = .TRUE.

~~~~~~~~~~~ Fan(FanNum)\%NVPerfNum = NVPerfNum

~~~~~~~~~~~ EXIT

~~~~~~~~~ END IF

~~~~~~~ END DO

~~~~~~~ IF ( .NOT. NVPerfFanFound) THEN

~~~~~~~~~ CALL ShowSevereError(TRIM(cCurrentModuleObject)//`, fan name not found ='//TRIM(cAlphaArgs(1)))

~~~~~~~~~ ErrorsFound = .true.

~~~~~~~ END IF

~~~~~ END DO

~~~~~ DEALLOCATE(cAlphaArgs)

~~~~~ DEALLOCATE(cAlphaFieldNames)

~~~~~ DEALLOCATE(lAlphaFieldBlanks)

~~~~~ DEALLOCATE(cNumericFieldNames)

~~~~~ DEALLOCATE(lNumericFieldBlanks)

~~~~~ DEALLOCATE(rNumericArgs)

~~~~~ IF (ErrorsFound) THEN

~~~~~~~~~~~ CALL ShowFatalError(RoutineName//`Errors found in input.~ Program terminates.')

~~~~~ ENDIF

~~~~~ Do FanNum = 1,NumFans

~~~~~~~~~~~~ ! Setup Report variables for the Fans

~~~~~~ CALL SetupOutputVariable(`Fan Electric Power{[}W{]}', Fan(FanNum)\%FanPower, `System',`Average',Fan(FanNum)\%FanName)

~~~~~~ CALL SetupOutputVariable(`Fan Delta Temp{[}C{]}', Fan(FanNum)\%DeltaTemp, `System',`Average',Fan(FanNum)\%FanName)

~~~~~~ CALL SetupOutputVariable(`Fan Electric Consumption{[}J{]}', Fan(FanNum)\%FanEnergy, `System',`Sum',Fan(FanNum)\%FanName, \&

~~~~~~~~~~~~~~~~~~~~~~~~~~~~~~~~ ResourceTypeKey = `Electric',GroupKey = `System', \&

~~ ~~~~~~~~~~~~~~~~~~~~~~~~~~~~~~EndUseKey = `Fans',EndUseSubKey = Fan(FanNum)\%EndUseSubcategoryName)

~~~~~ END DO

~~~~~ DO OnOffFanNum = 1,~ NumOnOff

~~~~~~ FanNum = NumSimpFan + NumVarVolFan + NumZoneExhFan + OnOffFanNum

~~~~~~ CALL SetupOutputVariable(`On/Off Fan Runtime Fraction', Fan(FanNum)\%FanRuntimeFraction, `System',`Average', \&

~~~~~~~~~~~~~~~~~~~~~~~~~~~~~~~~ Fan(FanNum)\%FanName)

~~~~~ END DO

~ RETURN

\textbf{END SUBROUTINE GetFanInput}

! End of Get Input subroutines for the HB Module

!******************************************************************************

! Beginning Initialization Section of the Module

!******************************************************************************

\textbf{SUBROUTINE InitFan(FanNum,FirstHVACIteration)}

~~~~~~~~~ ! SUBROUTINE INFORMATION:

~~~~~~~~~ !~~~~~~ AUTHOR~~~~~~~~ Richard J. Liesen

~~~~~~~~~ !~~~~~~ DATE WRITTEN~~ February 1998

~~~~~~~~~ !~~~~~~ MODIFIED~~~~~~ na

~~~~~~~~~ !~~~~~~ RE-ENGINEERED~ na

~~~~~~~~~ ! PURPOSE OF THIS SUBROUTINE:

~~~~~~~~~ ! This subroutine is for initializations of the Fan Components.

~~~~~~~~~ ! METHODOLOGY EMPLOYED:

~~~~~~~~~ ! Uses the status flags to trigger initializations.

~~~~~~~~~ ! REFERENCES:

~~~~~~~~~ ! na

~~~~~~~~~ ! USE STATEMENTS:

~ USE DataSizing, ONLY: CurSysNum

~ USE DataAirLoop, ONLY: AirLoopControlInfo

~ IMPLICIT NONE~~~ ! Enforce explicit typing of all variables in this routine

~~~~~~~~~ ! SUBROUTINE ARGUMENT DEFINITIONS:

~ LOGICAL, INTENT (IN):: FirstHVACIteration

~ Integer, Intent(IN) :: FanNum

~~~~~~~~~ ! SUBROUTINE PARAMETER DEFINITIONS:

~~~~~~~~~ ! na

~~~~~~~~~ ! INTERFACE BLOCK SPECIFICATIONS

~~~~~~~~~ ! na

~~~~~~~~~ ! DERIVED TYPE DEFINITIONS

~~~~~~~~~ ! na

~~~~~~~~~ ! SUBROUTINE LOCAL VARIABLE DECLARATIONS:

~ Integer~~~~~~~~~~~~ :: InletNode

~ Integer~~~~~~~~~~~~ :: OutletNode

~ Integer~~~~~~~~~~~~ :: InNode

~ Integer~~~~~~~~~~~~ :: OutNode

~ LOGICAL,SAVE~~~~~~~ :: MyOneTimeFlag = .true.

~ LOGICAL, ALLOCATABLE,Save, DIMENSION(:) :: MyEnvrnFlag

~ LOGICAL, ALLOCATABLE,Save, DIMENSION(:) :: MySizeFlag

~~~~~~~~~ ! FLOW:

~ IF (MyOneTimeFlag) THEN

~~~ ALLOCATE(MyEnvrnFlag(NumFans))

~~~ ALLOCATE(MySizeFlag(NumFans))

~~~ MyEnvrnFlag = .TRUE.

~~~ MySizeFlag = .TRUE.

~~~ MyOneTimeFlag = .false.

~ END IF

~ IF ( .NOT. SysSizingCalc .AND. MySizeFlag(FanNum)) THEN

~~~ CALL SizeFan(FanNum)

~~~ ! Set the loop cycling flag

~~~ IF (Fan(FanNum)\%Control = = `ONOFF') THEN

~~~~~ IF (CurSysNum \textgreater{} 0) THEN

~~~~~~~ AirLoopControlInfo(CurSysNum)\%CyclingFan = .TRUE.

~ ~~~~END IF

~~~ END IF

~~~ MySizeFlag(FanNum) = .FALSE.

~ END IF

! Do the Begin Environment initializations

~ IF (BeginEnvrnFlag .and. MyEnvrnFlag(FanNum)) THEN

~~~ !For all Fan inlet nodes convert the Volume flow to a mass flow

~~~ InNode = Fan(FanNum)\%InletNodeNum

~~~ OutNode = Fan(FanNum)\%OutletNodeNum

~~~ Fan(FanNum)\%RhoAirStdInit = PsyRhoAirFnPbTdbW(StdBaroPress,20.0,0.0)

~~~ !Change the Volume Flow Rates to Mass Flow Rates

~~~ Fan(FanNum)\%MaxAirMassFlowRate = Fan(FanNum)\%MaxAirFlowRate *~ Fan(FanNum)\%RhoAirStdInit

~~~ Fan(FanNum)\%MinAirMassFlowRate = Fan(FanNum)\%MinAirFlowRate *~ Fan(FanNum)\%RhoAirStdInit

~~~ !Init the Node Control variables

~~~ Node(OutNode)\%MassFlowRateMax~~~~~ = Fan(FanNum)\%MaxAirMassFlowRate

~~~ Node(OutNode)\%MassFlowRateMin~~~~~ = Fan(FanNum)\%MinAirMassFlowRate

~~~ !Initialize all report variables to a known state at beginning of simulation

~~~ Fan(FanNum)\%FanPower = 0.0

~~~ Fan(FanNum)\%DeltaTemp = 0.0

~~~ Fan(FanNum)\%FanEnergy = 0.0

~~~ MyEnvrnFlag(FanNum) = .FALSE.

~ END IF

~ IF (.not. BeginEnvrnFlag) THEN

~~~ MyEnvrnFlag(FanNum) = .true.

~ ENDIF

~ ! Do the Begin Day initializations

~~~ ! none

~ ! Do the begin HVAC time step initializations

~~~ ! none

~ ! Do the following initializations (every time step): This should be the info from

~ ! the previous components outlets or the node data in this section.

~ ! Do a check and make sure that the max and min available(control) flow is

~ ! between the physical max and min for the Fan while operating.

~ InletNode = Fan(FanNum)\%InletNodeNum

~ OutletNode = Fan(FanNum)\%OutletNodeNum

~ Fan(FanNum)\%MassFlowRateMaxAvail = MIN(Node(OutletNode)\%MassFlowRateMax, \&

~~~~~~~~~~~~~~~~~~~~~~~~~~~~~~~~~~~~~~~~~~~~~ Node(InletNode)\%MassFlowRateMaxAvail)

~ Fan(FanNum)\%MassFlowRateMinAvail = MIN(MAX(Node(OutletNode)\%MassFlowRateMin, \&

~~~~~~~~~~~~~~~~~~~~~~~~~~~~~~~~~~~~~~~~~~~~ Node(InletNode)\%MassFlowRateMinAvail), \&

~~~~~~~~~~~~~~~~~~~~~~~~~~~~~~~~~~~~~~~~~~~~ Node(InletNode)\%MassFlowRateMaxAvail)

~ ! Load the node data in this section for the component simulation

~ !

~ !First need to make sure that the massflowrate is between the max and min avail.

~ IF (Fan(FanNum)\%FanType .NE. `ZONE EXHAUST FAN') THEN

~~~ Fan(FanNum)\%InletAirMassFlowRate = Min(Node(InletNode)\%MassFlowRate, \&

~~~~~~~~~~~~~~~~~~~~~~~~~~~~~~~~~~~~~~~~~~ Fan(FanNum)\%MassFlowRateMaxAvail)

~~~ Fan(FanNum)\%InletAirMassFlowRate = Max(Fan(FanNum)\%InletAirMassFlowRate, \&

~~~~~~~~~~~~~~~~~~~~~~~~ ~~~~~~~~~~~~~~~~~~Fan(FanNum)\%MassFlowRateMinAvail)

~ ELSE~ ! zone exhaust fans - always run at the max

~~~ Fan(FanNum)\%MassFlowRateMaxAvail = Fan(FanNum)\%MaxAirMassFlowRate

~~~ Fan(FanNum)\%MassFlowRateMinAvail = 0.0

~~~ Fan(FanNum)\%InletAirMassFlowRate = Fan(FanNum)\%MassFlowRateMaxAvail

~ END IF

~ !Then set the other conditions

~ Fan(FanNum)\%InletAirTemp~~~~~~~~ = Node(InletNode)\%Temp

~ Fan(FanNum)\%InletAirHumRat~~~~~~ = Node(InletNode)\%HumRat

~ Fan(FanNum)\%InletAirEnthalpy~~~~ = Node(InletNode)\%Enthalpy

~ RETURN

\textbf{END SUBROUTINE InitFan}

\textbf{SUBROUTINE SizeFan(FanNum)}

~~~~~~~~~ ! SUBROUTINE INFORMATION:

~~~~~~~~~ !~~~~~~ AUTHOR~~~~~~~~ Fred Buhl

~~~~~~~~~ !~~~~~~ DATE WRITTEN~~ September 2001

~~~~~~~~~ !~~~~~~ MODIFIED~~~~~~ na

~~~~~~~~~ !~~~~~~ RE-ENGINEERED~ na

~~~~~~~~~ ! PURPOSE OF THIS SUBROUTINE:

~~~~~~~~~ ! This subroutine is for sizing Fan Components for which flow rates have not been

~~~~~~~~~ ! specified in the input.

~~~~~~~~~ ! METHODOLOGY EMPLOYED:

~~~~~~~~~ ! Obtains flow rates from the zone or system sizing arrays.

~~~~~~~~~ ! REFERENCES:

~~~~~~~~~ ! na

~~~~~~~~~ ! USE STATEMENTS:

~ USE DataSizing

~ IMPLICIT NONE~~~ ! Enforce explicit typing of all variables in this routine

~~~~~~~~~ ! SUBROUTINE ARGUMENT DEFINITIONS:

~ Integer, Intent(IN) :: FanNum

~~~~~~~~~ ! SUBROUTINE PARAMETER DEFINITIONS:

~~~~~~~~~ ! na

~~~~~~~~~ ! INTERFACE BLOCK SPECIFICATIONS

~~~~~~~~~ ! na

~~~~~~~~~ ! DERIVED TYPE DEFINITIONS

~~~~~~~~~ ! na

~~~~~~~~~ ! SUBROUTINE LOCAL VARIABLE DECLARATIONS:

~ REAL :: FanMinAirFlowRate

~ EXTERNAL ReportSizingOutput

~ FanMinAirFlowRate = 0.0

~ IF (Fan(FanNum)\%MaxAirFlowRate = = AutoSize) THEN

~~~ IF (CurSysNum \textgreater{} 0) THEN

~~~~~ CALL CheckSysSizing(`FAN:'//TRIM(Fan(FanNum)\%FanType)// `:' // TRIM(Fan(FanNum)\%Control), \&

~~~~~~~~~~~~~~~~~~~~~~~~~~ Fan(FanNum)\%FanName)

~~~~~ SELECT CASE(CurDuctType)

~~~~~~~ CASE(Main)

~~~~~~~~~ Fan(FanNum)\%MaxAirFlowRate = FinalSysSizing(CurSysNum)\%DesMainVolFlow

~~~~~~~~~ FanMinAirFlowRate = CalcSysSizing(CurSysNum)\%SysAirMinFlowRat * CalcSysSizing(CurSysNum)\%DesMainVolFlow

~~~~~~~ CASE(Cooling)

~~~~~~~~~ Fan(FanNum)\%MaxAirFlowRate = FinalSysSizing(CurSysNum)\%DesCoolVolFlow

~~~~~~~~~ FanMinAirFlowRate = CalcSysSizing(CurSysNum)\%SysAirMinFlowRat * CalcSysSizing(CurSysNum)\%DesCoolVolFlow

~~~~~~~ CASE(Heating)

~~~~~~~~~ Fan(FanNum)\%MaxAirFlowRate = FinalSysSizing(CurSysNum)\%DesHeatVolFlow

~~~~~~~~~ FanMinAirFlowRate = CalcSysSizing(CurSysNum)\%SysAirMinFlowRat * CalcSysSizing(CurSysNum)\%DesHeatVolFlow

~~~~~~~ CASE(Other)

~~~~~~~~~ Fan(FanNum)\%MaxAirFlowRate = FinalSysSizing(CurSysNum)\%DesMainVolFlow

~~~~~~~~~ FanMinAirFlowRate = CalcSysSizing(CurSysNum)\%SysAirMinFlowRat * CalcSysSizing(CurSysNum)\%DesMainVolFlow

~~~~~~~ CASE DEFAULT

~~~~~~~~~ Fan(FanNum)\%MaxAirFlowRate = FinalSysSizing(CurSysNum)\%DesMainVolFlow

~~~~~~~~~ FanMinAirFlowRate = CalcSysSizing(CurSysNum)\%SysAirMinFlowRat * CalcSysSizing(CurSysNum)\%DesMainVolFlow

~~~~~ END SELECT

~~~ ELSE IF (CurZoneEqNum \textgreater{} 0) THEN

~~~~~ CALL CheckZoneSizing(`FAN:' // TRIM(Fan(FanNum)\%FanType) // `:' // TRIM(Fan(FanNum)\%Control), \&

~~~~~~~~~~~~~~~~~~~~~~~~~~ Fan(FanNum)\%FanName)

~~~~~ IF (.NOT. ZoneHeatingOnlyFan) THEN

~~~~~~~ Fan(FanNum)\%MaxAirFlowRate = MAX(FinalZoneSizing(CurZoneEqNum)\%DesCoolVolFlow, \&

~~~~~~~~~~~~~~~~~~~~~~~~~~~~~~~~~~~~~~~~ FinalZoneSizing(CurZoneEqNum)\%DesHeatVolFlow)

~~~~~ ELSE

~~~~~~~ Fan(FanNum)\%MaxAirFlowRate = FinalZoneSizing(CurZoneEqNum)\%DesHeatVolFlow

~~~~~ END IF

~~~ END IF

~~~ IF (Fan(FanNum)\%MaxAirFlowRate \textless{} SmallAirVolFlow) THEN

~~~~~ Fan(FanNum)\%MaxAirFlowRate = 0.0

~~~ END IF

~~~ CALL ReportSizingOutput(`FAN:' // TRIM(Fan(FanNum)\%FanType) // `:' // TRIM(Fan(FanNum)\%Control), \&

~~~ ~~~~~~~~~~~~~~~~~~~~~~~~Fan(FanNum)\%FanName, `Max Flow Rate {[}m3/s{]}', Fan(FanNum)\%MaxAirFlowRate)

~~~ IF (Fan(FanNum)\%Control = = `VARIABLEVOLUME') THEN

~~~~~ CALL CheckSysSizing(`FAN:' // TRIM(Fan(FanNum)\%FanType) // `:' // TRIM(Fan(FanNum)\%Control), \&

~~ ~~~~~~~~~~~~~~~~~~~~~~~~Fan(FanNum)\%FanName)

~~~~~ Fan(FanNum)\%MinAirFlowRate = FanMinAirFlowRate

~~~~~ CALL ReportSizingOutput(`FAN:' // TRIM(Fan(FanNum)\%FanType) // `:' // TRIM(Fan(FanNum)\%Control), \&

~~~~~~~~~~~~~~~~~~~~~~~~~~~~~ Fan(FanNum)\%FanName, `Min Flow Rate {[}m3/s{]}', Fan(FanNum)\%MinAirFlowRate)

~~~ END IF

~ END IF

~ RETURN

\textbf{END SUBROUTINE SizeFan}

! End Initialization Section of the Module

!******************************************************************************

! Begin Algorithm Section of the Module

!******************************************************************************

\textbf{SUBROUTINE SimSimpleFan(FanNum)}

~~~~~~~~~ ! SUBROUTINE INFORMATION:

~~~~~~~~~ !~~~~~~ AUTHOR~~~~~~~~ Unknown

~~~~~~~~~ !~~~~~~ DATE WRITTEN~~ Unknown

~~~~~~~~~ !~~~~~~ MODIFIED~~~~~~ na

~~~~~~~~~ !~~~~~~ RE-ENGINEERED~ na

~~~~~~~~~ ! PURPOSE OF THIS SUBROUTINE:

~~~~~~~~~ ! This subroutine simulates the simple constant volume fan.

~~~~~~~~~ ! METHODOLOGY EMPLOYED:

~~~~~~~~~ ! Converts design pressure rise and efficiency into fan power and temperature rise

~~~~~~~~~ ! Constant fan pressure rise is assumed.

~~~~~~~~~ ! REFERENCES:

~~~~~~~~~ ! ASHRAE HVAC 2 Toolkit, page 2-3 (FANSIM)

~~~~~~~~~ ! USE STATEMENTS:

~~~~~~~~~ ! na

~ IMPLICIT NONE~~~ ! Enforce explicit typing of all variables in this routine

~~~~~~~~~ ! SUBROUTINE ARGUMENT DEFINITIONS:

~~ Integer, Intent(IN) :: FanNum

~~~~~~~~~ ! SUBROUTINE PARAMETER DEFINITIONS:

~~~~~~~~~ ! na

~~~~~~~~~ ! INTERFACE BLOCK SPECIFICATIONS

~~~~~~~~~ ! na

~~~~~~~~~ ! DERIVED TYPE DEFINITIONS

~~~~~~~~~ ! na

~~~~~~~~~ ! SUBROUTINE LOCAL VARIABLE DECLARATIONS:

~~~~~ Real RhoAir

~~~~~ Real DeltaPress~ ! {[}N/M\^{}2{]}

~~~~~ Real FanEff

~~~~~ Real MassFlow~~~ ! {[}kg/sec{]}

~~~~~ Real Tin~~~~~~~~ ! {[}C{]}

~~~~~ Real Win

~~~~~ Real FanShaftPower ! power delivered to fan shaft

~~~~~ Real PowerLossToAir ! fan and motor loss to air stream (watts)

~~ DeltaPress = Fan(FanNum)\%DeltaPress

~~ FanEff~~~~ = Fan(FanNum)\%FanEff

~~ ! For a Constant Volume Simple Fan the Max Flow Rate is the Flow Rate for the fan

~~ Tin~~~~~~~ = Fan(FanNum)\%InletAirTemp

~~ Win~~~~~~~ = Fan(FanNum)\%InletAirHumRat

~~ RhoAir~~~~ = Fan(FanNum)\%RhoAirStdInit

~~ MassFlow~~ = MIN(Fan(FanNum)\%InletAirMassFlowRate,Fan(FanNum)\%MaxAirMassFlowRate)

~~ MassFlow~~ = MAX(MassFlow,Fan(FanNum)\%MinAirMassFlowRate)

~~ !

~~ !Determine the Fan Schedule for the Time step

~ If( ( GetCurrentScheduleValue(Fan(FanNum)\%SchedPtr)\textgreater{}0.0 .and. Massflow\textgreater{}0.0 .or. TurnFansOn .and. Massflow\textgreater{}0.0) \&

~~~~~~~ .and. .NOT.TurnFansOff ) Then

~~ !Fan is operating

~~ Fan(FanNum)\%FanPower = MassFlow*DeltaPress/(FanEff*RhoAir) ! total fan power

~~ FanShaftPower = Fan(FanNum)\%MotEff * Fan(FanNum)\%FanPower~ ! power delivered to shaft

~~ PowerLossToAir = FanShaftPower + (Fan(FanNum)\%FanPower - FanShaftPower) * Fan(FanNum)\%MotInAirFrac

~~ Fan(FanNum)\%OutletAirEnthalpy = Fan(FanNum)\%InletAirEnthalpy + PowerLossToAir/MassFlow

~~ ! This fan does not change the moisture or Mass Flow across the component

~~ Fan(FanNum)\%OutletAirHumRat~~~~~~ = Fan(FanNum)\%InletAirHumRat

~~ Fan(FanNum)\%OutletAirMassFlowRate = MassFlow

~~ Fan(FanNum)\%OutletAirTemp = PsyTdbFnHW(Fan(FanNum)\%OutletAirEnthalpy,Fan(FanNum)\%OutletAirHumRat)

~Else

~~ !Fan is off and not operating no power consumed and mass flow rate.

~~ Fan(FanNum)\%FanPower = 0.0

~~ FanShaftPower = 0.0

~~ PowerLossToAir = 0.0

~~ Fan(FanNum)\%OutletAirMassFlowRate = 0.0

~~ Fan(FanNum)\%OutletAirHumRat~~~~~~ = Fan(FanNum)\%InletAirHumRat

~~ Fan(FanNum)\%OutletAirEnthalpy~~~~ = Fan(FanNum)\%InletAirEnthalpy

~~ Fan(FanNum)\%OutletAirTemp = Fan(FanNum)\%InletAirTemp

~~ ! Set the Control Flow variables to 0.0 flow when OFF.

~~ Fan(FanNum)\%MassFlowRateMaxAvail = 0.0

~~ Fan(FanNum)\%MassFlowRateMinAvail = 0.0

~End If

~RETURN

\textbf{END SUBROUTINE SimSimpleFan}

\textbf{SUBROUTINE SimVariableVolumeFan(FanNum)}

~~~~~~~~~ ! SUBROUTINE INFORMATION:

~~~~~~~~~ !~~~~~~ AUTHOR~~~~~~~~ Unknown

~~~~~~~~~ !~~~~~~ DATE WRITTEN~~ Unknown

~~~~~~ ~~~!~~~~~~ MODIFIED~~~~~~ Phil Haves

~~~~~~~~~ !~~~~~~ RE-ENGINEERED~ na

~~~~~~~~~ ! PURPOSE OF THIS SUBROUTINE:

~~~~~~~~~ ! This subroutine simulates the simple variable volume fan.

~~~~~~~~~ ! METHODOLOGY EMPLOYED:

~~~~~~~~~ ! Converts design pressure rise and efficiency into fan power and temperature rise

~~~~~~~~~ ! Constant fan pressure rise is assumed.

~~~~~~~~~ ! Uses curves of fan power fraction vs.~fan part load to determine fan power at

~~~~~~~~~ ! off design conditions.

~~~~~~~~~ ! REFERENCES:

~~~~~~~~~ ! ASHRAE HVAC 2 Toolkit, page 2-3 (FANSIM)

~~~~~~~~~ ! USE STATEMENTS:

~~~~~~~~~ ! na

~ IMPLICIT NONE~~~ ! Enforce explicit typing of all variables in this routine

~~~~~~~~~ ! SUBROUTINE ARGUMENT DEFINITIONS:

~~ Integer, Intent(IN) :: FanNum

~~~~~~~~~ ! SUBROUTINE PARAMETER DEFINITIONS:

~~~~~~~~~ ! na

~~~~~~~~~ ! INTERFACE BLOCK SPECIFICATIONS

~~~~~~~~~ ! na

~~~~~~~~~ ! DERIVED TYPE DEFINITIONS

~~~~~~~~~ ! na

~~~~~~~~~ ! SUBROUTINE LOCAL VARIABLE DECLARATIONS:

~~~~~ Real RhoAir

~~~~~ Real DeltaPress~ ! {[}N/M\^{}2 = Pa{]}

~~~~~ Real FanEff~~~~~ ! Total fan efficiency - combined efficiency of fan, drive train,

~~~~~~~~~~~~~~~~~~~~~~ ! motor and variable speed controller (if any)

~~~~~ Real MassFlow~~~ ! {[}kg/sec{]}

~~~~~ Real Tin~~~~~~~~ ! {[}C{]}

~~~~~ Real Win

~~~~~ Real PartLoadFrac

~~~~~ REAL MaxFlowFrac~~ !Variable Volume Fan Max Flow Fraction {[}-{]}

~~~~~ REAL MinFlowFrac~~ !Variable Volume Fan Min Flow Fraction {[}-{]}

~~~~~ REAL FlowFrac~~~~~ !Variable Volume Fan Flow Fraction {[}-{]}

~~~~~ Real FanShaftPower ! power delivered to fan shaft

~~~~~ Real PowerLossToAir ! fan and motor loss to air stream (watts)

! Simple Variable Volume Fan - default values from DOE-2

! Type of Fan~~~~~~~~~ Coeff1~~~~~~ Coeff2~~~~~~ Coeff3~~~~~~~ Coeff4~~~~~ Coeff5

! INLET VANE DAMPERS~~ 0.35071223~~ 0.30850535~~ -0.54137364~~ 0.87198823~ 0.000

! DISCHARGE DAMPERS~~~ 0.37073425~~ 0.97250253~~ -0.34240761~~ 0.000~~~~~~ 0.000

! VARIABLE SPEED MOTOR 0.0015302446 0.0052080574~ 1.1086242~~ -0.11635563~ 0.000

~~ DeltaPress~ = Fan(FanNum)\%DeltaPress

~~ FanEff~~~~~ = Fan(FanNum)\%FanEff

~~ Tin~~~~~~~~ = Fan(FanNum)\%InletAirTemp

~~ Win~~~~~~~~ = Fan(FanNum)\%InletAirHumRat

~~ RhoAir~~~~~ = Fan(FanNum)\%RhoAirStdInit

~~ MassFlow~~~ = MIN(Fan(FanNum)\%InletAirMassFlowRate,Fan(FanNum)\%MaxAirMassFlowRate)

~~ ! MassFlow~~~ = MAX(MassFlow,Fan(FanNum)\%MinAirMassFlowRate)

~ ! Calculate and check limits on fraction of system flow

~ MaxFlowFrac = 1.0

~ ! MinFlowFrac is calculated from the ration of the volume flows and is non-dimensional

~ MinFlowFrac = Fan(FanNum)\%MinAirFlowRate/Fan(FanNum)\%MaxAirFlowRate

~ ! The actual flow fraction is calculated from MassFlow and the MaxVolumeFlow * AirDensity

~ FlowFrac = MassFlow/(Fan(FanNum)\%MaxAirMassFlowRate)

! Calculate the part Load Fraction~~~~~~~~~~~~ (PH 7/13/03)

~ FlowFrac = MAX(MinFlowFrac,MIN(FlowFrac,1.0))~ ! limit flow fraction to allowed range

~ PartLoadFrac = Fan(FanNum)\%FanCoeff(1) + Fan(FanNum)\%FanCoeff(2)*FlowFrac +~ \&

~~~~~~~~~~~~~~~~~ Fan(FanNum)\%FanCoeff(3)*FlowFrac**2 + Fan(FanNum)\%FanCoeff(4)*FlowFrac**3 + \&

~~~~~~~~~~~~~~~~~ Fan(FanNum)\%FanCoeff(5)*FlowFrac**4

~~ !Determine the Fan Schedule for the Time step

~ If( ( GetCurrentScheduleValue(Fan(FanNum)\%SchedPtr)\textgreater{}0.0 .and. Massflow\textgreater{}0.0 .or. TurnFansOn .and. Massflow\textgreater{}0.0) \&

~~~~~~~ .and. .NOT.TurnFansOff ) Then

~~ !Fan is operating - calculate power loss and enthalpy rise

!~~ Fan(FanNum)\%FanPower = PartLoadFrac*FullMassFlow*DeltaPress/(FanEff*RhoAir) ! total fan power

~~ Fan(FanNum)\%FanPower = PartLoadFrac*Fan(FanNum)\%MaxAirMassFlowRate*DeltaPress/(FanEff*RhoAir) ! total fan power (PH 7/13/03)

~~ FanShaftPower = Fan(FanNum)\%MotEff * Fan(FanNum)\%FanPower~ ! power delivered to shaft

~~ PowerLossToAir = FanShaftPower + (Fan(FanNum)\%FanPower - FanShaftPower) * Fan(FanNum)\%MotInAirFrac

~~ Fan(FanNum)\%OutletAirEnthalpy = Fan(FanNum)\%InletAirEnthalpy + PowerLossToAir/MassFlow

~~ ! This fan does not change the moisture or Mass Flow across the component

~~ Fan(FanNum)\%OutletAirHumRat~~~~~~ = Fan(FanNum)\%InletAirHumRat

~~ Fan(FanNum)\%OutletAirMassFlowRate = MassFlow

~~ Fan(FanNum)\%OutletAirTemp = PsyTdbFnHW(Fan(FanNum)\%OutletAirEnthalpy,Fan(FanNum)\%OutletAirHumRat)

~ Else

~~ !Fan is off and not operating no power consumed and mass flow rate.

~~ Fan(FanNum)\%FanPower = 0.0

~~ FanShaftPower = 0.0

~~ PowerLossToAir = 0.0

~~ Fan(FanNum)\%OutletAirMassFlowRate = 0.0

~~ Fan(FanNum)\%OutletAirHumRat~~~~~~ = Fan(FanNum)\%InletAirHumRat

~~ Fan(FanNum)\%OutletAirEnthalpy~~~~ = Fan(FanNum)\%InletAirEnthalpy

~~ Fan(FanNum)\%OutletAirTemp = Fan(FanNum)\%InletAirTemp

~~ ! Set the Control Flow variables to 0.0 flow when OFF.

~~ Fan(FanNum)\%MassFlowRateMaxAvail = 0.0

~~ Fan(FanNum)\%MassFlowRateMinAvail = 0.0

~ End If

~ RETURN

\textbf{END SUBROUTINE SimVariableVolumeFan}

\textbf{SUBROUTINE SimOnOffFan(FanNum)}

~~~~~~~~~ ! SUBROUTINE INFORMATION:

~~~~~~~~~ !~~~~~~ AUTHOR~~~~~~~~ Unknown

~~~~~~~~~ !~~~~~~ DATE WRITTEN~~ Unknown

~~~~~~~~~ !~~~~~~ MODIFIED~~~~~~ Shirey, May 2001

~~~~~~~~~ !~~~~~~ RE-ENGINEERED~ na

~~~~~~~~~ ! PURPOSE OF THIS SUBROUTINE:

~~~~~~~~~ ! This subroutine simulates the simple on/off fan.

~~~~~~~~~ ! METHODOLOGY EMPLOYED:

~~~~~~~~~ ! Converts design pressure rise and efficiency into fan power and temperature rise

~~~~~~~~~ ! Constant fan pressure rise is assumed.

~~~~~~~~~ ! Uses curves of fan power fraction vs.~fan part load to determine fan power at

~~~~~~~~~ ! off design conditions.

~~~~~~~~~ ! Same as simple (constant volume) fan, except added part-load curve input

~~~~~~~~~ ! REFERENCES:

~~~~~~~~~ ! ASHRAE HVAC 2 Toolkit, page 2-3 (FANSIM)

~~~~~~~~~ ! USE STATEMENTS:

~ USE CurveManager, ONLY: CurveValue

~ IMPLICIT NONE~~~ ! Enforce explicit typing of all variables in this routine

~~~~~~~~~ ! SUBROUTINE ARGUMENT DEFINITIONS:

~~ Integer, Intent(IN) :: FanNum

~~~~~~~~~ ! SUBROUTINE PARAMETER DEFINITIONS:

~~~~~~~~~ ! na

~~~~~~~~~ ! INTERFACE BLOCK SPECIFICATIONS

~~~~~~~~~ ! na

~~~~~~~~~ ! DERIVED TYPE DEFINITIONS

~~~~~~~~~ ! na

~~~~~~~~~ ! SUBROUTINE LOCAL VARIABLE DECLARATIONS:

~~~~~ Real RhoAir

~~~~~ Real DeltaPress~ ! {[}N/M\^{}2{]}

~~~~~ Real FanEff

~~~~~ Real MassFlow~~~ ! {[}kg/sec{]}

~~~~~ Real Tin~~~~~~~~ ! {[}C{]}

~~~~~ Real Win

~~~~~ Real PartLoadRatio !Ratio of actual mass flow rate to max mass flow rate

~~~~~ REAL FlowFrac~~~~~ !Actual Fan Flow Fraction = actual mass flow rate / max air mass flow rate

~~~~~ Real FanShaftPower ! power delivered to fan shaft

~~~~~ Real PowerLossToAir ! fan and motor loss to air stream (watts)

~~ DeltaPress = Fan(FanNum)\%DeltaPress

~~ FanEff~~~~ = Fan(FanNum)\%FanEff

~~ Tin~~~~~~~ = Fan(FanNum)\%InletAirTemp

~~ Win~~~~~~~ = Fan(FanNum)\%InletAirHumRat

~~ RhoAir~~~~ = Fan(FanNum)\%RhoAirStdInit

~~ MassFlow~~ = MIN(Fan(FanNum)\%InletAirMassFlowRate,Fan(FanNum)\%MaxAirMassFlowRate)

~~ MassFlow~~ = MAX(MassFlow,Fan(FanNum)\%MinAirMassFlowRate)

~~ Fan(FanNum)\%FanRuntimeFraction = 0.0

~ ! The actual flow fraction is calculated from MassFlow and the MaxVolumeFlow * AirDensity

~ FlowFrac = MassFlow/(Fan(FanNum)\%MaxAirMassFlowRate)

~ ! Calculate the part load ratio, can't be greater than 1

~ PartLoadRatio = MIN(1.0,FlowFrac)

~ ! Determine the Fan Schedule for the Time step

~ IF( ( GetCurrentScheduleValue(Fan(FanNum)\%SchedPtr)\textgreater{}0.0 .and. Massflow\textgreater{}0.0 .or. TurnFansOn .and. Massflow\textgreater{}0.0) \&

~~~ ~~~~.and. .NOT.TurnFansOff ) THEN

~~ ! Fan is operating

~~ IF (OnOffFanPartLoadFraction \textless{} = 0.0) THEN

~~~~ CALL ShowWarningError(`FAN:SIMPLE:ONOFF, OnOffFanPartLoadFraction \textless{} = 0.0, Reset to 1.0')

~~~~ OnOffFanPartLoadFraction = 1.0 ! avoid divide by zero or negative PLF

~~ END IF

~~ IF (OnOffFanPartLoadFraction \textless{} 0.7) THEN

~~~~~~~ OnOffFanPartLoadFraction = 0.7 ! a warning message is already issued from the DX coils or gas heating coil

~~ END IF

~~ ! Keep fan runtime fraction between 0.0 and 1.0

~~ Fan(FanNum)\%FanRuntimeFraction = MAX(0.0,MIN(1.0,PartLoadRatio/OnOffFanPartLoadFraction))

~~ ! Fan(FanNum)\%FanPower = MassFlow*DeltaPress/(FanEff*RhoAir*OnOffFanPartLoadFraction)! total fan power

~~ Fan(FanNum)\%FanPower = Fan(FanNum)\%MaxAirMassFlowRate*Fan(FanNum)\%FanRuntimeFraction*DeltaPress/(FanEff*RhoAir)!total fan power

~~ ! OnOffFanPartLoadFraction is passed via DataHVACGlobals from the cooling or heating coil that is

~~ !~~ requesting the fan to operate in cycling fan/cycling coil mode

~~ OnOffFanPartLoadFraction = 1.0 ! reset to 1 in case other on/off fan is called without a part load curve

~~ FanShaftPower = Fan(FanNum)\%MotEff * Fan(FanNum)\%FanPower~ ! power delivered to shaft

~~ PowerLossToAir = FanShaftPower + (Fan(FanNum)\%FanPower - FanShaftPower) * Fan(FanNum)\%MotInAirFrac

~~ Fan(FanNum)\%OutletAirEnthalpy = Fan(FanNum)\%InletAirEnthalpy + PowerLossToAir/MassFlow

~~ ! This fan does not change the moisture or Mass Flow across the component

~~ Fan(FanNum)\%OutletAirHumRat~~~~~~ = Fan(FanNum)\%InletAirHumRat

~~ Fan(FanNum)\%OutletAirMassFlowRate = MassFlow

!~~ Fan(FanNum)\%OutletAirTemp = Tin + PowerLossToAir/(MassFlow*PsyCpAirFnW(Win))

~~ Fan(FanNum)\%OutletAirTemp = PsyTdbFnHW(Fan(FanNum)\%OutletAirEnthalpy,Fan(FanNum)\%OutletAirHumRat)

~ ELSE

~~ ! Fan is off and not operating no power consumed and mass flow rate.

~~ Fan(FanNum)\%FanPower = 0.0

~~ FanShaftPower = 0.0

~~ PowerLossToAir = 0.0

~~ Fan(FanNum)\%OutletAirMassFlowRate = 0.0

~~ Fan(FanNum)\%OutletAirHumRat~~~~~~ = Fan(FanNum)\%InletAirHumRat

~~ Fan(FanNum)\%OutletAirEnthalpy~~~~ = Fan(FanNum)\%InletAirEnthalpy

~~ Fan(FanNum)\%OutletAirTemp = Fan(FanNum)\%InletAirTemp

~~ ! Set the Control Flow variables to 0.0 flow when OFF.

~~ Fan(FanNum)\%MassFlowRateMaxAvail = 0.0

~~ Fan(FanNum)\%MassFlowRateMinAvail = 0.0

~ END IF

~ RETURN

\textbf{END SUBROUTINE SimOnOffFan}

\textbf{SUBROUTINE SimZoneExhaustFan(FanNum)}

~~~~~~~~~ ! SUBROUTINE INFORMATION:

~~~~~~~~~ !~~~~~~ AUTHOR~~~~~~~~ Fred Buhl

~~~~~~~~~ !~~~~~~ DATE WRITTEN~~ Jan 2000

~~~~~~~~~ !~~~~~~ MODIFIED~~~~~~ na

~~~~~~~~~ !~~~~~~ RE-ENGINEERED~ na

~~~~~~~~~ ! PURPOSE OF THIS SUBROUTINE:

~~~~~~~~~ ! This subroutine simulates the Zone Exhaust Fan

~~~~~~~~~ ! METHODOLOGY EMPLOYED:

~~~~~~~~~ ! Converts design pressure rise and efficiency into fan power and temperature rise

~~~~~~~~~ ! Constant fan pressure rise is assumed.

~~~~~~~~~ ! REFERENCES:

~~~~~~~~~ ! ASHRAE HVAC 2 Toolkit, page 2-3 (FANSIM)

~~~~~~~~~ ! USE STATEMENTS:

~~~~~~~~~ ! na

~ IMPLICIT NONE~~~ ! Enforce explicit typing of all variables in this routine

~~~~~~~~~ ! SUBROUTINE ARGUMENT DEFINITIONS:

~~ Integer, Intent(IN) :: FanNum

~~~~~~~~~ ! SUBROUTINE PARAMETER DEFINITIONS:

~~~~~~~~~ ! na

~~~~~~~~~ ! INTERFACE BLOCK SPECIFICATIONS

~~~~~~~~~ ! na

~~~~~~~~~ ! DERIVED TYPE DEFINITIONS

~~~~~~~~~ ! na

~~~~~~~~~ ! SUBROUTINE LOCAL VARIABLE DECLARATIONS:

~~~~~ Real RhoAir

~~~~~ Real DeltaPress~ ! {[}N/M\^{}2{]}

~~~~~ Real FanEff

~~~~~ Real MassFlow~~~ ! {[}kg/sec{]}

~~~~~ Real Tin~~~~~~~~ ! {[}C{]}

~~~~~ Real Win

~~~ ~~Real PowerLossToAir ! fan and motor loss to air stream (watts)

~~ DeltaPress = Fan(FanNum)\%DeltaPress

~~ FanEff~~~~ = Fan(FanNum)\%FanEff

~~ ! For a Constant Volume Simple Fan the Max Flow Rate is the Flow Rate for the fan

~~ Tin~~~~~~~ = Fan(FanNum)\%InletAirTemp

~~ Win~~~~~~~ = Fan(FanNum)\%InletAirHumRat

~~ RhoAir~~~~ = Fan(FanNum)\%RhoAirStdInit

~~ MassFlow~~ = Fan(FanNum)\%InletAirMassFlowRate

~~ !

~~ !Determine the Fan Schedule for the Time step

~ If( ( GetCurrentScheduleValue(Fan(FanNum)\%SchedPtr)\textgreater{}0.0 .or. TurnFansOn ) \&

~~~~~~~ .and. .NOT.TurnFansOff ) Then

~~ !Fan is operating

~~ Fan(FanNum)\%FanPower = MassFlow*DeltaPress/(FanEff*RhoAir) ! total fan power

~~ PowerLossToAir = Fan(FanNum)\%FanPower

~~ Fan(FanNum)\%OutletAirEnthalpy = Fan(FanNum)\%InletAirEnthalpy + PowerLossToAir/MassFlow

~~ ! This fan does not change the moisture or Mass Flow across the component

~~ Fan(FanNum)\%OutletAirHumRat~~~~~~ = Fan(FanNum)\%InletAirHumRat

~~ Fan(FanNum)\%OutletAirMassFlowRate = MassFlow

~~ Fan(FanNum)\%OutletAirTemp = PsyTdbFnHW(Fan(FanNum)\%OutletAirEnthalpy,Fan(FanNum)\%OutletAirHumRat)

~Else

~~ !Fan is off and not operating no power consumed and mass flow rate.

~~ Fan(FanNum)\%FanPower = 0.0

~~ PowerLossToAir = 0.0

~~ Fan(FanNum)\%OutletAirMassFlowRate = 0.0

~~ Fan(FanNum)\%OutletAirHumRat~~~~~~ = Fan(FanNum)\%InletAirHumRat

~~ Fan(FanNum)\%OutletAirEnthalpy~~~~ = Fan(FanNum)\%InletAirEnthalpy

~~ Fan(FanNum)\%OutletAirTemp = Fan(FanNum)\%InletAirTemp

~~ ! Set the Control Flow variables to 0.0 flow when OFF.

~~ Fan(FanNum)\%MassFlowRateMaxAvail = 0.0

~~ Fan(FanNum)\%MassFlowRateMinAvail = 0.0

~~ Fan(FanNum)\%InletAirMassFlowRate = 0.0

~End If

~RETURN

\textbf{END SUBROUTINE SimZoneExhaustFan}

! End Algorithm Section of the Module

! *****************************************************************************

! Beginning of Update subroutines for the Fan Module

! *****************************************************************************

\textbf{SUBROUTINE UpdateFan(FanNum)}

~~~~~~~~~ ! SUBROUTINE INFORMATION:

~~~~~~~~~ !~~~~~~ AUTHOR~~~~~~~~ Richard Liesen

~~~~~~~~~ !~~~~~~ DATE WRITTEN~~ April 1998

~~~~~~~~~ !~~~~~~ MODIFIED~~~~~~ na

~~~~~~~~~ !~~~~~~ RE-ENGINEERED~ na

~~~~~~~~~ ! PURPOSE OF THIS SUBROUTINE:

~~~~~~~~~ ! This subroutine updates the fan outlet nodes.

~~~~~~~~~ ! METHODOLOGY EMPLOYED:

~~~~~~~~~ ! Data is moved from the fan data structure to the fan outlet nodes.

~~~~~~~~~ ! REFERENCES:

~~~~~~~~~ ! na

~~~~~~~~~ ! USE STATEMENTS:

~~~~~~~~~ ! na

~ IMPLICIT NONE~~~ ! Enforce explicit typing of all variables in this routine

~~~~~~~~~ ! SUBROUTINE ARGUMENT DEFINITIONS:

~~ Integer, Intent(IN) :: FanNum

~~~~~~~~~ ! SUBROUTINE PARAMETER DEFINITIONS:

~~~~~~~~~ ! na

~~~~~~~~~ ! INTERFACE BLOCK SPECIFICATIONS

~~~~~~~~~ ! na

~~~~~~~~~ ! DERIVED TYPE DEFINITIONS

~~~~~~~~~ ! na

~~~~~~~~~ ! SUBROUTINE LOCAL VARIABLE DECLARATIONS:

~ Integer~~~~~~~~~~~~ :: OutletNode

~ Integer~~~~~~~~~~~~ :: InletNode

~~ OutletNode = Fan(FanNum)\%OutletNodeNum

~~ InletNode = Fan(FanNum)\%InletNodeNum

~~ ! Set the outlet air nodes of the fan

~~ Node(OutletNode)\%MassFlowRate~ = Fan(FanNum)\%OutletAirMassFlowRate

~~ Node(OutletNode)\%Temp~~~~~~~~~ = Fan(FanNum)\%OutletAirTemp

~~ Node(OutletNode)\%HumRat~~~~~~~ = Fan(FanNum)\%OutletAirHumRat

~~ Node(OutletNode)\%Enthalpy~~~~~ = Fan(FanNum)\%OutletAirEnthalpy

~~ ! Set the outlet nodes for properties that just pass through \& not used

~~ Node(OutletNode)\%Quality~~~~~~~~ = Node(InletNode)\%Quality

~~ Node(OutletNode)\%Press~~~~~~~~~~ = Node(InletNode)\%Press

~~ ! Set the Node Flow Control Variables from the Fan Control Variables

~~ Node(OutletNode)\%MassFlowRateMaxAvail = Fan(FanNum)\%MassFlowRateMaxAvail

~~ Node(OutletNode)\%MassFlowRateMinAvail = Fan(FanNum)\%MassFlowRateMinAvail

~~ IF (Fan(FanNum)\%FanType .EQ. `ZONE EXHAUST FAN') THEN

~~~~ Node(InletNode)\%MassFlowRate = Fan(FanNum)\%InletAirMassFlowRate

~~ END IF

~ RETURN

\textbf{END Subroutine UpdateFan}

!~~~~~~~ End of Update subroutines for the Fan Module

! *****************************************************************************

! Beginning of Reporting subroutines for the Fan Module

! *****************************************************************************

\textbf{SUBROUTINE ReportFan(FanNum)}

~~~~~~~~~ ! SUBROUTINE INFORMATION:

~~~~~~~~~ !~~~~~~ AUTHOR~~~~~~~~ Richard Liesen

~~~~~~~~~ !~~~~~~ DATE WRITTEN~~ April 1998

~~~~~~~~~ !~~~~~~ MODIFIED~~~~~~ na

~~~~~~~~~ !~~~~~~ RE-ENGINEERED~ na

~~~~~~~~~ ! PURPOSE OF THIS SUBROUTINE:

~~~~~~~~~ ! This subroutine updates the report variables for the fans.

~~~~~~~~~ ! METHODOLOGY EMPLOYED:

~~~~~~~~~ ! na

~~~~~~~~~ ! REFERENCES:

~~~~~~~~~ ! na

~~~~~~~~~ ! USE STATEMENTS:

~ Use DataHVACGlobals, ONLY: TimeStepSys, FanElecPower

~ IMPLICIT NONE~~~ ! Enforce explicit typing of all variables in this routine

~~~~~~~~~ ! SUBROUTINE ARGUMENT DEFINITIONS:

~~ Integer, Intent(IN) :: FanNum

~~~~~~~~~ ! SUBROUTINE PARAMETER DEFINITIONS:

~~~~~~~~~ ! na

~~~~~~~~~ ! INTERFACE BLOCK SPECIFICATIONS

~~~~~~~~~ ! na

~~~~~~~~~ ! DERIVED TYPE DEFINITIONS

~~~~~~~~~ ! na

~~~~~~~~~ ! SUBROUTINE LOCAL VARIABLE DECLARATIONS:

~~~~~~~~~ ! na

~~~ Fan(FanNum)\%FanEnergy = Fan(FanNum)\%FanPower*TimeStepSys*3600

~~~ Fan(FanNum)\%DeltaTemp = Fan(FanNum)\%OutletAirTemp - Fan(FanNum)\%InletAirTemp

~~~ FanElecPower = Fan(FanNum)\%FanPower

~ RETURN

\textbf{END Subroutine ReportFan}

!~~~~~~~ End of Reporting subroutines for the Fan Module

! *****************************************************************************

! Beginning of Utility subroutines for the Fan Module

! *****************************************************************************

\textbf{FUNCTION GetFanDesignVolumeFlowRate(FanType,FanName,ErrorsFound) RESULT(DesignVolumeFlowRate)}

~~~~~~~~~ ! FUNCTION INFORMATION:

~~~~ ~~~~~!~~~~~~ AUTHOR~~~~~~~~ Linda Lawrie

~~~~~~~~~ !~~~~~~ DATE WRITTEN~~ February 2006

~~~~~~~~~ !~~~~~~ MODIFIED~~~~~~ na

~~~~~~~~~ !~~~~~~ RE-ENGINEERED~ na

~~~~~~~~~ ! PURPOSE OF THIS FUNCTION:

~~~~~~~~~ ! This function looks up the design volume flow rate for the given fan and returns it.~ If

~~~~~~~~~ ! incorrect fan type or name is given, errorsfound is returned as true and value is returned

~~~~~~~~~ ! as negative.

~~~~~~~~~ ! METHODOLOGY EMPLOYED:

~~~~~~~~~ ! na

~~~~~~~~~ ! REFERENCES:

~~~~~~~~ ~! na

~~~~~~~~~ ! USE STATEMENTS:

~ USE InputProcessor,~ ONLY: FindItemInList

~ IMPLICIT NONE ! Enforce explicit typing of all variables in this routine

~~~~~~~~~ ! FUNCTION ARGUMENT DEFINITIONS:

~ CHARACTER(len = *), INTENT(IN) :: FanType~~~~~ ! must match fan types in this module

~ CHARACTER(len = *), INTENT(IN) :: FanName~~~~~ ! must match fan names for the fan type

~ LOGICAL, INTENT(INOUT)~~~~~~ :: ErrorsFound~ ! set to true if problem

~ REAL~~~~~~~~~~~~~~~~~~~~~~~~ :: DesignVolumeFlowRate ! returned flow rate of matched fan

~~~~~~~~~ ! FUNCTION PARAMETER DEFINITIONS:

~~~~~~~~~ ! na

~~~~~~~~~ ! INTERFACE BLOCK SPECIFICATIONS:

~~~~~~~~~ ! na

~~~~~~~~~ ! DERIVED TYPE DEFINITIONS:

~~~~~ ~~~~! na

~~~~~~~~~ ! FUNCTION LOCAL VARIABLE DECLARATIONS:

~ INTEGER :: WhichFan

~ ! Obtains and Allocates fan related parameters from input file

~ IF (GetFanInputFlag) THEN~ !First time subroutine has been entered

~~~ CALL GetFanInput

~~~ GetFanInputFlag = .false.

~ End If

~ WhichFan = FindItemInList(FanName,Fan\%FanName,NumFans)

~ IF (WhichFan / = 0) THEN

~~~ DesignVolumeFlowRate = Fan(WhichFan)\%MaxAirFlowRate

~ ENDIF

~ IF (WhichFan = = 0) THEN

~~~ CALL ShowSevereError(`Could not find FanType = ``'//TRIM(FanType)//''' with Name = ``'//TRIM(FanName)//'''\,')

~~~ ErrorsFound = .true.

~~~ DesignVolumeFlowRate = -1000.

~ ENDIF

~ RETURN

\textbf{END FUNCTION GetFanDesignVolumeFlowRate}

\textbf{FUNCTION GetFanInletNode(FanType,FanName,ErrorsFound) RESULT(NodeNumber)}

~~~~~~~~~ ! FUNCTION INFORMATION:

~~~~~~~~~ !~~~~~~ AUTHOR~~~~~~~~ Linda Lawrie

~~~~~~~~~ !~~~~~~ DATE WRITTEN~~ February 2006

~~~~~~~~~ !~~~~~~ MODIFIED~~~~~~ na

~~~~~~~~~ !~~~~~~ RE-ENGINEERED~ na

~~~~~~~~~ ! PURPOSE OF THIS FUNCTION:

~~~~~~~~~ ! This function looks up the given fan and returns the inlet node.~ If

~~~~~~~~~ ! incorrect fan type or name is given, errorsfound is returned as true and value is returned

~~~~~~~~~ ! as zero.

~~~~~~~~~ ! METHODOLOGY EMPLOYED:

~~~~~~~~~ ! na

~~~~~~~~~ ! REFERENCES:

~~~~~~~~~ ! na

~~~~~~~~~ ! USE STATEMENTS:

~ USE InputProcessor,~ ONLY: FindItemInList

~ IMPLICIT NONE ! Enforce explicit typing of all variables in this routine

~~~~~~~~~ ! FUNCTION ARGUMENT DEFINITIONS:

~ CHARACTER(len = *), INTENT(IN) :: FanType~~~~~ ! must match fan types in this module

~ CHARACTER(len = *), INTENT(IN) :: FanName~~~~~ ! must match fan names for the fan type

~ LOGICAL, INTENT(INOUT)~~~~~~ :: ErrorsFound~ ! set to true if problem

~ INTEGER~~ ~~~~~~~~~~~~~~~~~~~:: NodeNumber~~ ! returned outlet node of matched fan

~~~~~~~~~ ! FUNCTION PARAMETER DEFINITIONS:

~~~~~~~~~ ! na

~~~~~~~~~ ! INTERFACE BLOCK SPECIFICATIONS:

~~~~~~~~~ ! na

~~~~~~~~~ ! DERIVED TYPE DEFINITIONS:

~~~~~~~~~ ! na

~~~~~~~ ~~! FUNCTION LOCAL VARIABLE DECLARATIONS:

~ INTEGER :: WhichFan

~ ! Obtains and Allocates fan related parameters from input file

~ IF (GetFanInputFlag) THEN~ !First time subroutine has been entered

~~~ CALL GetFanInput

~~~ GetFanInputFlag = .false.

~ End If

~ WhichFan = FindItemInList(FanName,Fan\%FanName,NumFans)

~ IF (WhichFan / = 0) THEN

~~~ NodeNumber = Fan(WhichFan)\%InletNodeNum

~ ENDIF

~ IF (WhichFan = = 0) THEN

~~~ CALL ShowSevereError(`Could not find FanType = ``'//TRIM(FanType)//''' with Name = ``'//TRIM(FanName)//'''\,')

~~~ ErrorsFound = .true.

~~~ NodeNumber = 0

~ ENDIF

~ RETURN

\textbf{END FUNCTION GetFanInletNode}

\textbf{FUNCTION GetFanOutletNode(FanType,FanName,ErrorsFound) RESULT(NodeNumber)}

~~~~~~~~~ ! FUNCTION INFORMATION:

~~~~~~~~~ !~~~~~~ AUTHOR~~~~~~~~ Linda Lawrie

~~~~~~~~~ !~~~~~~ DATE WRITTEN~~ February 2006

~~~~~~~~~ !~~~~~~ MODIFIED~~~~~~ na

~~~~~~~~~ !~~~~~~ RE-ENGINEERED~ na

~~~~~~~~~ ! PURPOSE OF THIS FUNCTION:

~~~~ ~~~~~! This function looks up the given fan and returns the outlet node.~ If

~~~~~~~~~ ! incorrect fan type or name is given, errorsfound is returned as true and value is returned

~~~~~~~~~ ! as zero.

~~~~~~~~~ ! METHODOLOGY EMPLOYED:

~~~~~~~~~ ! na

~~~ ~~~~~~! REFERENCES:

~~~~~~~~~ ! na

~~~~~~~~~ ! USE STATEMENTS:

~ USE InputProcessor,~ ONLY: FindItemInList

~ IMPLICIT NONE ! Enforce explicit typing of all variables in this routine

~~~~~~~~~ ! FUNCTION ARGUMENT DEFINITIONS:

~ CHARACTER(len = *), INTENT(IN) :: FanType~~~~~ ! must match fan types in this module

~ CHARACTER(len = *), INTENT(IN) :: FanName~~~~~ ! must match fan names for the fan type

~ LOGICAL, INTENT(INOUT)~~~~~~ :: ErrorsFound~ ! set to true if problem

~ INTEGER~~ ~~~~~~~~~~~~~~~~~~~:: NodeNumber~~ ! returned outlet node of matched fan

~~~~~~~~~ ! FUNCTION PARAMETER DEFINITIONS:

~~~~~~~~~ ! na

~~~~~~~~~ ! INTERFACE BLOCK SPECIFICATIONS:

~~~~~~~~~ ! na

~~~~~~~~~ ! DERIVED TYPE DEFINITIONS:

~~~~~~~~~ ! na

~~~~~~~ ~~! FUNCTION LOCAL VARIABLE DECLARATIONS:

~ INTEGER :: WhichFan

~ ! Obtains and Allocates fan related parameters from input file

~ IF (GetFanInputFlag) THEN~ !First time subroutine has been entered

~~~ CALL GetFanInput

~~~ GetFanInputFlag = .false.

~ End If

~ WhichFan = FindItemInList(FanName,Fan\%FanName,NumFans)

~ IF (WhichFan / = 0) THEN

~~~ NodeNumber = Fan(WhichFan)\%OutletNodeNum

~ ENDIF

~ IF (WhichFan = = 0) THEN

~~~ CALL ShowSevereError(`Could not find FanType = ``'//TRIM(FanType)//''' with Name = ``'//TRIM(FanName)//'''\,')

~~~ ErrorsFound = .true.

~~~ NodeNumber = 0

~ ENDIF

~ RETURN

\textbf{END FUNCTION GetFanOutletNode}

! End of Utility subroutines for the Fan Module

! *****************************************************************************

!~~~~ NOTICE

!

!~~~~ Copyright © 1996-xxxx The Board of Trustees of the University of Illinois

!~~~~ and The Regents of the University of California through Ernest Orlando Lawrence

!~~~~ Berkeley National Laboratory.~ All rights reserved.

!

!~~~~ Portions of the EnergyPlus software package have been developed and copyrighted

!~~~~ by other individuals, companies and institutions.~ These portions have been

!~~~~ incorporated into the EnergyPlus software package under license.~~ For a complete

!~~~~ list of contributors, see ``Notice'' located in EnergyPlus.f90.

!

!~~~~ NOTICE: The U.S. Government is granted for itself and others acting on its

!~~~~ behalf a paid-up, nonexclusive, irrevocable, worldwide license in this data to

!~~~~ reproduce, prepare derivative works, and perform publicly and display publicly.

!~~~~ Beginning five (5) years after permission to assert copyright is granted,

!~~~~ subject to two possible five year renewals, the U.S. Government is granted for

!~~~~ itself and others acting on its behalf a paid-up, non-exclusive, irrevocable

!~~~~ worldwide license in this data to reproduce, prepare derivative works,

!~~~~ distribute copies to the public, perform publicly and display publicly, and to

!~~~~ permit others to do so.

!

!~~~~ TRADEMARKS: EnergyPlus is a trademark of the US Department of Energy.

!

\textbf{End Module Fans}
