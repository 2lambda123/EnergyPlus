\section{HVAC Sizing Tips}\label{hvac-sizing-tips}

To help achieve successful autosizing of HVAC equipment, note the following general guidelines.

\begin{itemize}
\item
  Begin with everything fully autosized (no user-specified values) and get a working system before trying to control any specific sized.
\item
  The user must coordinate system controls with sizing inputs. For example, if the Sizing:System ``Central Cooling Design Supply Air Temperature'' is set to 13C, the user must make sure that the setpoint manager for the central cooling coil controls to 13C as design conditions. EnergyPlus does not cross-check these inputs. The sizing calculations use the information in the Sizing:* objects. The simulation uses the information in controllers and setpoint managers.
\item
  User-specified flow rates will only impact the sizing calculations if entered in the Sizing:Zone or Sizing:System objects. Sizing information flows only from the sizing objects to the components. The sizing calculations have no knowledge of user-specified values in a component. The only exception to this rule is that plant loop sizing will collect all component design water flow rates whether autosized or user-specified.
\item
  The zone thermostat schedules determine the times at which design loads will be calculated. All zone-level schedules (such as lights, electric equipment, infiltration) are active during the sizing calculations (using the day type specified for the sizing period). System and plant schedules (such as availability managers and component schedules) are unknown to the sizing calculations. To exclude certain times of day from the sizing load calculations, use the thermostat setpoint schedules for SummerDesignDay and/or WinterDesignDay. For example, setting the cooling setpoint schedule to 99C during nighttime hours for the SummerDesignDay day type will turn off cooling during those hours.
\end{itemize}

For more information, read the Input Output Reference section on ``Input for Design Calculations and Component Autosizing.''
