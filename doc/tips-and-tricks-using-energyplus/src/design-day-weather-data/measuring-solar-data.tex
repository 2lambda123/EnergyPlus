\section{Measuring Solar Data}\label{measuring-solar-data}

\emph{Can the following weather file metrics be directly measured by some inexpensive devices?}

Extraterrestrial Horizontal Radiation \{Wh/m2\} Extraterrestrial Direct Normal Radiation \{Wh/m2\} Horizontal Infrared Radiation Intensity from Sky \{Wh/m2\} Global Horizontal Radiation \{Wh/m2\} Direct Normal Radiation \{Wh/m2\} Diffuse Horizontal Radiation \{Wh/m2\} Global Horizontal Illuminance \{lux\} Direct Normal Illuminance \{lux\} Diffuse Horizontal Illuminance \{lux\}*

You can't measure extraterrestrial unless you're in outer space, but then it's assumed to be constant anyway. For the various radiation and illuminance values, they can measured by various instrumentation ranging from the very cheap to the very expensive. Properly, radiation needs to be measured with a pyranometer (Eppley), which is pricy, but I'm also seen people use simpler apparatus (Lycors) that are really photometers. Direct beam is generally not measured, but derived by subtracting the diffuse from the global. Diffuse is measured by adding a shadow band over a pyranometer to block out the direct beam. Pyranometers measure heat, photometers measure light. All the illuminance on the weather files are derived from the radiation and sky conditions.

Do not forget that the quantities you list are the inputs to the models that are used to derive the variables you really need in practice: irradiance and illuminance on the facets of the building (windows especially). These facets are usually NOT horizontal. Measuring all the components for all tilts and azimuths can be a costly proposition, and that's why it is rarely done (hence the need for models), but that's what should be done in serious experiments to remove the (large) uncertainties in modeled radiation.

Illuminance is measured with photometers (from, e.g., Licor), which resemble silicon-based pyranometers. Both are less costly than thermopile radiometers, which are normally the best in terms of accuracy. Measurements obtained with silicon-based pyranometers need various corrections to account for their limited spectral range. No correction is needed for photometers, though. So you have this issue of accuracy vs cost to consider.

Direct irradiance is measured with a pyrheliometer, which tracks the sun and is therefore costly, but also the most accurate of all radiometers. Obtaining direct irradiance by subtracting diffuse from global is convenient, but not accurate, as shown in recent publications.
